\chapter{FIXME: Vymysliet nazov}

FIXME: uvod

\begin{defn}
    Nech $G$ je graf s vrcholmi usporiadanými v postupnosti $v_1,v_2, \ldots, v_n$.
    Nech $f$ je čiastočné $L(2,1)$-farbenie množiny $S$ v grafe $G$ s rozsahom $k$. Pod pojmom \emph{charakteristika rádu $k$
    zobrazenia $f$} budeme rozumieť reťazec $r \in \{0, 1, \bar{1}\}^n$, ktorý spĺňa:
    \begin{align*}
        r_i = 0 & \Leftrightarrow v_i \notin S \\
        r_i = 1 & \Leftrightarrow v_i \in S \wedge f(v_i) \neq k \\
        r_i = \bar{1} & \Leftrightarrow v_i \in S \wedge f(v_i) = k
    \end{align*}

    Ďalej pod pojmom \emph{rozšírená charakteristika rádu $k$ zobrazenia $f$} budeme rozumieť reťazec
    $s \in \{0, \bar{0}, 1, \bar{1}\}^n$, ktorý spĺňa
    \begin{align*}
        s_i = 0 & \Leftrightarrow r_i = 0 \wedge (\forall j \in \{1 \dots n\}: v_j \in N(v_i) \Rightarrow t_j \neq 1 \\
        s_i = \bar{0} & \Leftrightarrow r_i = 0 \wedge (\exists j \in \{1 \dots n\}: v_j \in N(v_i) \wedge t_j = 1 \\
        s_i = r_i & \Leftrightarrow r_i \neq 0
    \end{align*}

    Charakteristiky čiastočných farbení budeme ďalej v texte značiť $r$.
\end{defn}

Z algoritmu Junosza-Szaniawski môžeme vidieť, že na reprezentáciu množiny čiastočných farbení
s obmedzeným rozsahom nám stačí ukladať množinu charakteristík fixného rádu. Keď nepotrebujeme
pre každý ofarbený vrchol ukladať jeho presnú farbu, môžeme si dovoliť reprezentovať aj
vrcholové separátory a použiť ich na rozdelenie problému. O presnom spôsobe a aplikácií
takéhoto rozdelenia si povieme v ďalšej časti.

Predtým si ale stručne pripomenieme lemu o hranových separátoroch z predošlej kapitoly \ref{hrsep-lema}.
Táto lema hovorí, že existencia $L(2,1)$-farbenia grafu $G$ vyplýva z existencie $L(2,1)$-farbení
komponentov grafu $G-E$ pre hranový oddeľovač $E$. Tieto farbenia musia navyše farbiť všetky vrcholy
z hranového oddeľovača a musia sa zhodovať v ich farbách.

Podobné tvrdenie platí aj pre čiastočné farbenia. V tomto prípade je podmienkou, že všetky čiastočné
farbenia farbia tú istú podmnožinu vrcholov z hranového oddeľovača a navyše im priradzujú rovnakú farbu.
Keďže ide o takmer identické tvrdenie k leme o hranových separátoroch \ref{hrsep-lema}, nebudeme
jeho presné znenie uvádzať.

\section{Rozdelenie na vrcholových separátoroch}

Základnou myšlienkou rozdelenia na vrcholových separátoroch je, že ak máme zafixované čiastočné
farbenie separátora a všetkých jeho susedov, farbenie zvyšných vrcholov v rôznych komponentoch
je nezávislé.

\begin{lema}
    Nech $S_V$ je vrcholový separátor grafu $G$, nech $G_1, G_2, \ldots G_k$ sú komponenty
    grafu $G - S_V$ a nech $H_1, H_2, \ldots, H_k$ sú grafy indukované množinami $V(G_1) \cup S_V,\ V(G_2) \cup S_V,\ \ldots,\ V(G_k) \cup S_V$.
    Nech $f$ je čiastočné $L(2,1)$-farbenie nejakej podmnožiny $S \subseteq N[S_V]$ v grafe $G$ a nech $f_1, f_2, \ldots f_k$
    sú čiastočné farbenia množín $S_1, S_2, \ldots, S_k$ v grafoch $H_1, H_2, \ldots, H_k$ (v tomto poradí), ktoré spĺňajú podmienku

    $$ \forall i \in \{1 \ldots k\},\ \forall v \in V(H_i) \cap N[S_V]: (v \in S_i \Leftrightarrow v \in S) \wedge (v \in S_i \Rightarrow f_i(v) = f(v)).$$

    Potom zobrazenie $\omega: \bigcup \limits_{i=1}^k S_i \to \mathbb{N}$ definované nasledovne:

    \[ \omega(v) =
    \begin{cases}
        f(v), & v \in S \\
        f_i(v), & v \in S_i - S
    \end{cases}
    \]

    tvorí čiastočné farbenie grafu $G$.
\end{lema}

\begin{proof}
    Dôkaz je veľmi podobný dôkazu lemy o hranových separátoroch \ref{hrsep-lema} v predošlej kapitole. Všetky
    vrcholy ofarbené vrámci jedného grafu $H_i$ nemôžu porušiť podmienku čiastočného $L(2,1)$-farbenia. Podobne
    pre dva vrcholy v $S$.
    
    Každá dvojica vrcholov, kde jeden patrí do $S_i - S$ a druhý do $S - S_i$ má vzdialenosť
    aspoň $3$, lebo cesta medzi nimi musí obsahovať aspoň jeden úsek, ktorý tvorí hrana z $S_i - S$ do $N(S)$, hrana
    z $N(S)$ do $S$ a hrana z $S$ do iného vrchola $N(S)$.

    Nakoniec každá dvojica vrcholov, kde jeden patrí do $S_i - S$ a druhý do $S_j - S$, kde $i \neq j$, musí mať tiež
    vzdialenosť aspoň $3$.
\end{proof}

Okolie jedného vrchola môže byť pomerne veľké, napríklad vo hviezdach v ňom môžu byť všetky zvyšné vrcholy.
Pri reprezentácii čiastočného farbenia cez jeho charakteristiku nám to však nevadí. Rôznych charakteristík,
ktoré môžu mať čiastočné farbenia malého separátora a jeho okolia, je malý. Horný odhad pre počet rôznych
charakteristík si zhrnieme v nasledujúcej leme.

\begin{lema}
    Nech $S$ je vrcholový separátor grafu $G$, nech $q = |S|$ a $p = |N[S]|$, nech $k \in \mathbb{N}$ je ľubovoľné
    číslo, nech $F$ je množina všetkých čiastočných farbení s rozsahom $k$, ktoré ofarbujú nejakú podmnožinu $N[S]$,
    nech $R$ je množina všetkých charakteristík rádu $k$ farbení v $F$. Potom $|R| \leq 2^p \cdot (p+1)^{q}$.
\end{lema}
\begin{proof}
    Najprv zhora odhadneme počet rôznych množín vrcholov, ktoré majú priradenú farbu $k$. Pre každý vrchol
    $v \in S$ platí, že v jeho okolí $N[v]$ je nanajvýš jeden vrchol, ktorý má farbu $k$. V opačnom prípade by totiž
    mali dva vrcholy vo vzdialenosti nanajvýš $2$ priradenú tú istú farbu. Keďže celkový počet vrcholov
    v $N[S]$ je $p$, pre každý z $q$ vrcholov separátora môžeme vybrať nanajvýš jednu z $p$ možností, alebo
    nevybrať žiadny vrchol. Preto množín vrcholov, ktoré majú priradenú farbu $k$, je nanajvýš $(p+1)^q$.

    Ďalej pre každú množinu vrcholou s farbou $k$ ostáva nanajvýš $p$ vrcholov, každý z nich nezávisle
    patrí, alebo nepatrí do množiny ofarbených vrcholov. Pre každú možnosť množiny vrcholov s farbou $k$
    máme teda nanajvýš $2^p$ možností, ako vyzerá množina ostatných ofarbených vrcholov. Charakteristík
    je teda nanajvýš $2^p (p+1)^q$.
\end{proof}

Rozdeľovanie pomocou vrcholového separátora budeme využívať na konštrukciu rýchlejších algoritmov
pre problém $L(2,1)$-farbenia. Pre všetky možné charakteristiky separátora a okolia si budeme pamätať,
aké charakteristiky môžu mať farbenia vo všetkých komponentoch.

Podobne bude prebiehať pridanie novej
farby. Vyskúšame pridanie všetkých farieb do separátora a jeho okolia. Pre každú takúto možnosť potom
pridáme novú farbu do každého komponentu. Ako to bude presne vyzerať s postupom a s časovou zložitosťou,
si ukážeme na príklade planárnych grafov.

\section{Farbenie planárnych grafov}

Dôležitými výsledkami pre farbenie planárnych grafov sú vety o separátoroch v planárnych grafoch \cite{tarjan_plansep},
ktoré si uvedieme.

\begin{veta}
    \label{planarsep-veta}
    Nech $G$ je planárny graf s $n$ vrcholmi. Potom sa dajú vrcholy grafu $G$ rozdeliť do troch množín $A, B, C$ tak,
    že neexistuje hrana medzi žiadnym vrcholom v $A$ a žiadnym vrcholom v $B$, množiny $A$ aj $B$ majú nanajvýš $\frac{n}{2}$
    vrcholov a množina $C$ má nanajvýš $\frac{2 \sqrt{2n}}{1 - \sqrt{\frac{2}{3}}}$ vrcholov.
\end{veta}

\begin{pozn}
Hodnota výrazu $\frac{2 \sqrt{2}}{1 - \sqrt{\frac{2}{3}}}$ je menšia ako $16$, preto budeme kvôli prehľadnosti
veľkosť množiny $C$ zhora odhadovať ako $16 \sqrt{n}$.
\end{pozn}

Ďalej budeme vytvárať algoritmus, ktorý dokáže pre danú množinu charakteristík čiastočných $L(2,1)$-farbení
s rozsahom $k$ vypočítať množinu charakteristík čiastočných $L(2,1)$-farbení s rozsahom $k+1$. Túto funkciu
nad množinami charakteristík budeme ďalej označovať $\oplus$. Pre množinu charakteristík $R$ zadefinujeme
množinu suffixov reťazca $w$ ako $R_w = \{u | wu \in R \}$.

Základný postup pri počítaní $\oplus(R)$ s využitím separátorov bude nasledovný:
Vrcholy separátora a okolia (množinu $N[C]$, kde $C$ je množina z vety \ref{planarsep-veta})
označíme $S$, počet vrcholov v $S$ označíme $p$. Vrcholy grafu usporiadame tak, aby vrcholy $S$ boli na začiatku.
Množinu (nerozšírených) charakteristík $R$ rozdelíme podľa ohodnotenia $S$. Pre každú možnosť ohodnotenia separátora
a všetky možnosti ohodnotenia $S$ novou farbou nezávisle vypočítame všetky charakteristiky, ktoré môže dosahovať
zvyšok vrcholov. Nakoniec tieto čiastočné výsledky pospájame a odstránime dupllikáty.
Formálny zápis výpočtu a jeho odvodenia bude vyzerať nasledovne:

$$ \oplus(R) = \bigcup \limits_{w \in R} \oplus(w) = \bigcup \limits_{w \in \{0, 1, \bar{1}\}^p} \oplus(R_w) = \bigcup \limits_{u \in \{0, 1, \bar{1}\}^p} \left(\bigcup \limits_{\substack{w \in \{0, 1, \bar{1}\}^p\\ \textrm{t.ž. } u \in \oplus(w)}} \oplus(R_w) \right)$$

Riešenie
problému $L(2,1)$ farbenia potom niekoľkokrát aplikuje tento algoritmus, až kým nebude v množine
charakteristík $R$ taká, ktorá zodpovedá $L(2,1)$-farbeniu. Keďže každý graf má triviálne farbenie
s rozsahom $2n - 2$, tento algoritmus nám zaručene stačí spustiť najviac $(2n - 2)$-krát.

Základom algoritmu bude nájsť separátor podľa vety o planárnom separátore \ref{planarsep-veta}. Tento
krok stačí vykonať iba raz, preto si môžeme dovoliť jeho hľadanie tak, že prejdeme všetkých $2^n$ podmnožín
vrcholov a pre každú v polynomiálnom čase overíme, či je dostatočne malá a či po jej odstránení bude mať
každý komponent súvislosti nanajvýš $\frac{n}{2}$ vrcholov.

Ďalej teda predpokladáme, že poznáme vrcholový separátor $C$. Podľa toho, koľko vrcholov je v okolí separátora, čiže
v $N[C]$, budeme robiť rôzne veci. Najprv rozoberieme jednoduchší prípad, kde $|N[C]| \ge \frac{3n}{4}$,
potom rozoberieme prípad $|N[C]| < \frac{3n}{4}$.

\subsection{Separátor s veľkým okolím}

Popíšeme algoritmus, ktorý pre danú množinu charakteristík $R$ rádu $k$ spočíta novú množinu charakteristík
rádu $k+1$. Základný postup sme už popísali, pre každú možnosť starej a novej charakteristiky separátora
vyskúšame pridať novú farbu zvyšným vrcholom. Separátor označíme $S$, počet vrcholov v $N[S]$ označíme $p$, počet
vrcholov $S$ označíme $q$.

Pre ľubovoľnú charakteristiku separátora a jeho okolia existuje najviac $(p+1)^q$ možností, ktoré
vrcholy dostanú novú farbu $k+1$. Taktiež vieme všetky tieto možnosti generovať -- pre každý vrchol v separátore
vyberieme niektorý z vrcholov jeho uzavretého okolia, alebo nevyberieme v jeho okolí žiadny vrchol. Pre
každý takýto výber následne overíme, či sme nevybrali dva vrcholy vo vzdialenosti menšej, ako $3$, alebo
či nejaký vybratý vrchol nesusedí s vrcholom farby $k$.

Pre každú možnosť starej a novej charakteristiky separátora ostane nejaka množina charakteristík
zvyšných $n-p \leq \frac{n}{4}$ vrcholov, pre ktorú potrebujeme vypočítať nové charakteristiky. Tu si môžeme
dovoliť extrémne neefektívne riešenie, kde pre každú starú charakteristiku $r$ a každú podmnožinu $U$
jej neofarbených vrcholov vyskúšame, či je povolené ofarbiť všetky vrcholy $U$ farbou $k+1$. Takýmto
spôsobom budeme overovať nanajvýš $4^{n-p}$ možností -- keby boli vrcholy mimo separátora nezávislé,
pre každý máme iba $4$ typy možností, v ktorých ho budeme skúšať: Neofarbený vrchol ostáva neofarbený,
neofarbený vrchol farbíme $k+1$, vrchol ofarbený farbou $k$, vrchol ofarbený inou farbou.

Overenie
jedného výberu je polynomiálne, lebo stačí pre každú dvojicu vrcholov vo vzdialenosti nanajvýš
$2$ skontrolovať, či by ich priradenie farieb neporušilo podmienky $L(2,1)$-farbenia.

Pre každú z nanajvýš $2^p (p+1)^q$ možností ofarbenia $S$ vyskúšame nanajvýš $(p+1)^q$ možností pridania
novej farby do $S$. Pre každú takúto možnosť následne vyskúšame nanajvýš $4^{n-p}$ možností pridania novej
farby do množiny charakteristík a pre každú možnosť vykonáme polynomiálne veľa práce.

Celková časová
zložitosť teda bude $O^*(2^p (p+1)^{2q} 4^{n-p})$. Z vety o separátore v planárnych grafoch \ref{planarsep-veta}
vieme, že veľkosť separátora, $q$, je nanajvýš $16\sqrt{n}$. Túto vetvu algoritmu spúšťame, keď $p \ge \frac{3n}{4}$.
Funkcia $2^p 4^{n-p}$ klesá s $p$ a teda jeho najväčšia možná hodnota pre minimálne $p = \frac{3n}{4}$ je
$2^{\frac{5n}{4}}$. Výraz $(p+1)^{2q}$ zhora odhadneme na $(n+1)^{32\sqrt{n}} \leq (2n)^{32\sqrt{n}}$.

Dosadením horných odhadov dostaneme časovú zložitosť $O^*(2^{\frac{5n}{4} + 32\sqrt{n} + 32\lg{n}\sqrt{n}}) = O^*(2.38^{n + o(n)})$.

\subsection{Separátor s malým okolím}

V prípade separátora s malým okolím budeme využívať rozdelenie pomocou separátora. Nech $A, B, C$ sú množiny
vrcholov spĺňajúce vetu o planárnom separátore \ref{planarsep-veta}. Bez ujmy na všeobecnosti má množina $C$
viac susedov v množine $A$, než v množine $B$. Keďže dokopy má množina $N[C]$ menej ako $\frac{3n}{4}$ vrcholov
a množiny $A$ a $B$ sú disjunktné, množina $B \cap N[C]$ má nanajvýš polovicu z tohto počtu, čiže $\frac{3n}{8}$.

Pre konštrukciu algoritmu použijeme hranový separátor medzi množinami $B$ a $C$, ktorý obsahuje vrcholy
$C \cup (B \cap N[C])$. Túto množinu vrcholov budeme ďalej označovať $S$. Množinu hrán medzi množinami
$B$ a $C$ označíme $E$. Komponenty súvislosti grafu $G-E$ označíme $K_1, \ldots, K_l$. Počet vrcholov v $S$
označíme $p$, počet vrcholov v $C$ označíme $q$. FIXME: ďalšie označenia

Z lemy o hranových separátoroch vieme, že pre zafixovanú charakteristiku vrcholov $S$ môžeme pre každý
z komponentov $K_1 \ldots K_l$ ukladať množinu jeho charakteristík nezávisle. Množinu charakteristík
celého grafu by sme vedeli zostrojiť ako karteziánsky súčin týchto dielčích charakteristík, zjednotený cez
všetky charakteristiky množiny $S$.

Postup počítania funkcie $\oplus$ bude podobný, ako pri separátore s veľkým okolím. Pre každú možnosť
charakteristiky rádu $k$ vyskúšame všetky možnosti ofarbenia separátora $S$ farbou $k+1$. Pre každú
možnosť starej charakteristiky, novej charakteristiky a komponentu $K_i$ skonštruujeme z jeho pôvodnej
množiny charakteristík $R_i$ novú množinu charakteristík $\oplus(R_i)$. Keďže každý komponent tvorí súvislý
graf, môžeme na počítanie $\oplus(R_i)$ použiť postup z algoritmu Junosza-Szaniawski.

Pre počítanie operácie $\oplus$ podľa algoritmu Junosza-Szaniawski potrebujeme poznať rozšírené charakteristiky.
Pre každý komponent a každú jeho (nerozšírenú) charakteristiku vieme vypočítať jej rozšírený ekvivalent v
polynomiálnom čase: Pre každý vrchol s hodnotou $\bar{1}$ prečíslujeme všetkých susedov s hodnotou $0$
na hodnotu $\bar{0}$.

FIXME: Note o tom, ze cisto z algoritmu J-S nemame zarucenu konzistentnost s celym separatorom. Tuto
spravime ako postprocessing.

Všetky komponenty, ktoré obsahujú nejaký vrchol v $B$, majú nanajvýš $\frac{n}{2}$ vrcholov, lebo
oddeľovač $E$ oddeľuje množinu $B$ a $C$. Množiny $A$ a $B$ sú oddelené podľa vety o separátore
a množina $B$ má nanajvýš $\frac{n}{2}$ vrcholov.

Podobne, všetky komponenty, ktoré obsahujú nejaký vrchol v $A$, majú nanajvýš $\frac{n}{2} + 16 \sqrt{n}$
vrcholov, lebo množiny $A$ a $C$ majú dohromady najviac $\frac{n}{2} + 16 \sqrt{n}$ vrcholov a ich vrcholy
nie sú spojené s vrcholmi v $B$.

Počet charakteristík $S$ odhadneme podobne, ako pre vrcholové separátory: Pre každý vrchol v $C$ máme
nanajvýš $p+1$ možností, ako vybrať nanajvýš jeden vrchol z jeho okolia. Výberov pre všetky vrcholy je
teda nanajvýš $(p+1)^q$ a pre zvyšné vrcholy máme nanajvýš $2^p$ možností. Dohromady dostávame horný
odhad $2^p (p+1)^q$ možností. Pre každú z týchto možností potom máme nanajvýš $(p+1)^q$ možností, ktoré
vrcholy dostanú farbu $k+1$.

Nakoniec pre každú možnosť starej charakteristiky a množiny vrcholov s novou farbou budeme počítať
na každom komponente operáciu $\oplus$ pomocou algoritmu z článku Junosza-Szaniawskeho a kol. \cite{junosza_fast},
ktorá na komponente s veľkosťou $x$ pracuje v čase $O^*(2.65^x)$. Každý komponent, na ktorom ju budeme
spúšťať, má nanajvýš $\frac{n}{2} + 16\sqrt{n}$ vrcholov. Komponentov je nanajvýš $n$, teda budeme
potrebovať nanajvýš $n$ nezávislých výpočtov. Tento faktor sa stratí v $O^*$ notácii.

Časová zložitosť tohto algoritmu teda bude $O^*(2^{\frac{3n}{8}} \cdot (\frac{3n}{8} + 16 \sqrt{n} + 1)^{32 \sqrt{n}} \cdot 2.65^{\frac{n}{2} + 16 \sqrt{n}}) = O^*(2.12^{n + o(n)})$.

\subsection{Vylepšenie a poznámky}

Popísali sme si algoritmus, ktorý na základe veľkosti okolia separátora robil veľmi rôzne výpočty.
Pri vysvetľovaní sme ako hranicu použili ľahko zapísateľnú, ale nie najoptimálnejšiu konštantu
$\alpha = \frac{3}{4}$. Pre grafy, v ktorých má okolie separátora aspoň $\alpha n$ vrcholov
sme používali algoritmus pre separátor s veľkým okolím, pre ostatné algoritmus pre separátor
s malým okolím. Ďalej odhadneme lepšiu konštantu a časovú zložitosť, ktorá z jej použitia vyplýva.

Časová zložitosť algoritmu pre separátor s veľkým okolím je pre ľubovoľnú konštantu $\alpha \in (0, 1)$
v triede funkcií $O^*(2^{\alpha n + o(n)}4^{(1 - \alpha) n} = O^*(2^{n(2 - \alpha) + o(n)})$.

Časová zložitosť algoritmu pre separátor s malým okolím je pre ľubovoľnú konštantu $\alpha \in (0,1)$
v triede funkcií $O^*(2^{n\frac{\alpha}{2} + o(n)} 2.65^{\frac{n}{2} + o(n)})$.

Zvyšovaním konštanty $\alpha$ teda zlepšujeme časovú zložitosť algoritmu pre separátor s malým
okolím a zhoršujeme časovú zložitosť algoritmu pre separátor s veľkým okolím. Vyváženú časovú
zložitosť dostaneme pre takú konštantu $\alpha$, kde $2 - \alpha = \frac{\alpha}{2} + \frac{\lg(2.65)}{2}$.
Vyriešením tejto rovnice dostávame hodnotu $\alpha \approx 0.865$, z ktorej vyplýva časová zložitosť
$O^*(2.2^{n + o(n)})$ pre oba prípady.

Uvedený algoritmus sa okrem existencie vrcholového rozdeľovača veľkosti $O(\sqrt{n})$ nespoliehal
na žiadnu inú vlastnosť planárnych grafov. Preto uvedený algoritmus funguje pre všetky triedy grafov,
v ktorých existuje vrcholový separátor veľkosti $O(\sqrt{n})$. Toto tvrdenie vieme dokonca mierne
zosilniť na triedy grafov, v ktorých existuje vrcholový separátor veľkosti $O(n^{1 - \epsilon})$ pre
ľubovoľnú hodnotu $\epsilon > 0$.

FIXME: Napisat nejaky rozumny pseudokod. Ak ostane na taketo pompeznosti cas.

\section{Farbenie vyvážene rozdeliteľných grafov}

Na príkade planárnych grafov sme ukázali spôsob, ako využiť existenciu dostatočne malého vrcholového
separátora pre konštrukciu rýchlejšieho algoritmu na hľadanie minimálneho rozsahu $L(2,1)$-farbenia
daného grafu. Už pri tejto konštrukcii sme potrebovali prejsť od vrcholového separátora k hranovému
separátoru, ktorý má menej rôznych charakteristík. Ďalej túto myšlienku využijeme na konštrukciu
algoritmu pre dobre rozdeliteľné grafy.

\begin{defn}
    Nech $G$ je jednoduchý graf s $n$ vrcholmi, nech $A, B$ sú podmnožiny $V(G)$,
    nech $C = N(B)$ a $D = N(A)$. Dvojicu množín $(A,B)$ budeme volať \emph{vyvážené rozdelenie
    grafu $G$}, ak platí:
    \begin{enumerate}
        \item $A \cup B = V(G) \wedge A \cap B = \emptyset$
        \item $|A| \leq \frac{2n}{3}$
        \item $|B| \leq \frac{2n}{3}$
        \item $|C \cup D| \leq \frac{n}{4}$
    \end{enumerate}

    Triedu grafov, v ktorých existuje vyvážené rozdelenie, budeme volať \emph{vyvážene rozdeliteľné}.
\end{defn}

Pre ľubovoľný graf vieme overiť, či je vyvážene rozdeliteľný, v čase $O^*(2^n)$ a v prípade, že je
vyvážene rozdeliteľný aj nájdeme jeho vyvážené rozdelenie. Pre každú z $2^n$ podmnožín vrcholov
v polynomiálnom čase overíme, či môže byť množinou vo vyváženom rozdelení.

Pre ľubovoľnú podmnožinu $X \subset V(G)$ naozaj vieme
v polynomiálnom čase overiť, či môže byť množinou vyváženého rozdelenia: Skontrolujeme, či má
nanajvýš $\frac{2n}{3}$ a aspoň $\lceil \frac{n}{3} \rceil$ vrcholov. Ďalej pre každý vrchol v $X$
overíme, či nejaký z jeho susedov leží mimo $X$ a počet takýchto vrcholov označíme $c$. Podobne
pre každý vrchol mimo $X$ overíme, či má suseda v $X$ a počet takýchto vrcholov označíme $d$. Ak
platí $c + d \leq \frac{n}{4}$, tak dvojica $(X, V(G) - X)$ tvorí vyvážené rozdelenie grafu $G$.

Algoritmus pre hľadanie farbiaceho čísla vyvážene rozdeliteľných grafov budeme konštruovať podobne, ako
algoritmus pre planárne grafy (s malým okolím separátora). Najprv si označíme objekty, s ktorými budeme
pracovať. Graf budeme tradične označovať $G$, jeho počet vrcholov $n$. Množiny vyváženého rozdelenia $G$
označíme $A$ a $B$. Množinu $N(B)$ označíme $C$ a množinu $N(A)$ označíme $D$. Do pozornosti dávame, že
platí $N(B) \subset A$ a $N(A) \subset B$. Separátor bude tvorený vrcholmi $C \cup D$ a zodpovedá hranovému
separátoru $\{(u,v) \in E(G) | u \in C \wedge v \in D\}$, ktorý označíme $E$. Komponenty grafu $G - E$ budeme
označovať $K_1 \ldots K_m$.

Základný postup bude nasledovný: Pre každú možnosť charakteristiky separátora $C \cup D$,
každú možnosť pridania novej farby do $C \cup D$ a každý komponent (nezávisle) vypočítame, aká
množina charakteristík v ňom môže byť po pridaní novej farby. Pre tento výpočet budeme používať
postup z algoritmu Junosza-Szaniawski.

FIXME: Opat poznamka o tom, ze Junosza-Szaniawski nebude pouzity na cely graf, od ktoreho zavisi
a mnoziny budu prechadzat post-processingom.

Ďalej budeme odhadovať časovú zložitosť tohto algoritmu. Najzložitejšia časť odhadu sa bude týkať
počtu rôznych prechodov v separátore, čiže počtu kombinácií starej a novej charakteristiky separátora.
Najprv dokážeme, že separátor vieme pokryť hviezdami. Potom odvodíme počet prechodov v hviezde
danej veľkosti a nakoniec tento počet zhora odhadneme jednoduchšou funkciou.

\begin{defn}
    Graf $G$ voláme \emph{hviezda}, ak v ňom existuje vrchol $v$, ktorý je spojený hranou s každým iným
    vrcholom, a každá hrana v $G$ je incidentná s vrcholom $v$. Vrchol $v$ spĺňajúci tieto podmienky
    budeme volať \emph{stred hviezdy}.

    Graf $G$ voláme \emph{netriviálna hviezda}, ak má aspoň dva vrcholy a je hviezda.
\end{defn}

\begin{lema}
    Nech $G$ je súvislý graf s aspoň dvomi vrcholmi. Potom existuje množina jeho podgrafov
    $H_1, H_2, \ldots H_k$, ktorá spĺňa nasledujúce podmienky:
    \begin{enumerate}
        \item $\forall i, j \in \{1 \ldots k\}, i \neq j: V(H_i) \cap V(H_j) = \emptyset$,
        \item $\forall v \in V(G) \exists i \in \{1 \ldots k\}: v \in V(H_i)$,
        \item $\forall i \in \{1 \ldots k\}: H_i \textrm{ je netriviálna hviezda}$.
    \end{enumerate}
\end{lema}

\begin{proof}
    Dokážeme indukciou vzhľadom na počet vrcholov grafu, $n$.

    Báza indukcie, $n = 2$: Existuje iba jeden dvojvrcholový súvislý
    graf: obsahuje dva vrcholy prepojené hranou. Ľahko vidíme, že tento graf je hviezda a oba vrcholy
    môžu byť jej stredom.

    Indukčný krok, $n \to n+1$: V každom súvislom grafe $H$ existuje vrchol $u$ taký, že graf $H - u$ je súvislý.
    Nech $v$ je takýto vrchol v našom grafe $G$ a nech $H_1 \ldots H_k$ sú podgrafy $G-v$, ktoré spĺňajú
    podmienky lemy a ktorých existencia vyplýva z indukčného predpokladu. Keďže $G$ je súvislý graf s aspoň
    tromi vrcholmi, musí v ňom existovať aspoň jedna hrana $e$ medzi vrcholom $v$ a nejakým iným vrcholom $u$.
    Bez ujmy na všeobecnosti platí $u \in V(H_1)$.

    Ďalej rozoberieme niekoľko možností podľa toho, ako vyzerá hviezda $H_1$:

    \begin{description}
        \item[$H_1$ má dva vrcholy:] V tomto prípade môže byť stredom hviezdy ľubovoľný z vrcholov $H_1$.
        Preto graf $H_1 \cup \{e, v\}$ je hviezda s vrcholom $u$. Vyhovujúcou množinou pre lemu je teda
        $H_1 \cup \{e, v\}, H_2, \ldots, H_k$.
        \item[$H_1$ má aspoň tri vrcholy a $u$ je jeho stred:] Pokiaľ pripojíme k stredu hviezdy ďalší
        vrchol, dostaneme opäť hviezdu, preto $H_1 \cup \{e, v\}, H_2, \ldots H_k$ je množina hviezd
        spĺňajúca lemu.
        \item[$H_1$ má aspoň tri vrcholy a $u$ nie je jeho stred:] Keď z hviezdy odoberieme vrchol, ktorý
        nie je jej stredom, dostaneme opäť hviezdu. Keďže $H_1$ má aspoň tri vrcholy, $H_1 - u$ je netriviálna
        hviezda. Zároveň graf $H$ tvorený vrcholmi $u, v$ a hranou $e$ je netriviálna hviezda. Preto množina
        hviezd $H_1 - u, H_2, \ldots, H_k, H$ je množina hviezd spĺňajúca podmienky lemy.
    \end{description}
\end{proof}

Ďalej budeme odvádzať vzťah pre počet možných prechodov medzi charakteristikami hviezd v závislosti od veľkosti hviezdy.
Počet možných prechodov vieme zhora odhadnúť ako súčin medzi počtami pre jednotlivé hviezdy. Preto nás bude pre hviezdu
s $h$ vrcholmi zaujímať hlavne $h$-ta odmnocnina z jeho počtu prechodov. Uvidíme, že najvyššiu odmocninu budú
mať práve dvojvrcholové hviezdy. Najprv vypíšeme všetky prechody pre dvojvrcholové hviezdy a potom dokážeme všeobecný
vzťah pre hviezdy s viacerými vrcholmi.

Dvojvrcholová hviezda je tvorená dvomi vrcholmi, ktoré sú prepojené hranou. Charakteristiky budeme pre prehľadnosť
vypisovať ako dvojice čísel.
\begin{align*}
    (0, 0) \to & (0, 0) \ | \ (0, \bar{1}) \ | \ (\bar{1}, 0) \\
    (0, 1) \to & (0, 1) \ | \ (\bar{1}, 1) \\
    (0, \bar{1}) \to & (0, 1) \\
    (1, 0) \to & (1, 0) \ | \ (1, \bar{1}) \\
    (1, 1) \to & (1, 1) \\
    (1, \bar{1}) \to & (1, 1) \\
    (\bar{1}, 0) \to & (1, 0) \\
    (\bar{1}, 1) \to & (1, 1)
\end{align*}
