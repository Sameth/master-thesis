\chapter{FIXME: Vymysliet nazov}

FIXME: uvod

\begin{defn}
    Nech $f$ je čiastočné $L(2,1)$-farbenie množiny $S$ v grafe $G$ s rozsahom $k$. Pod pojmom \emph{charakteristika rádu $k$
    zobrazenia $f$} budeme rozumieť zobrazenie $c_{S,f,k}: V(G) \to \{0, 1, \bar{1}\}$ definované nasledovne:

    \[ c_{S,f,k} (v) =
    \begin{cases}
        0, & v \notin S \\
        1, & v \in S \wedge f(v) \neq k \\
        \bar{1}, & v \in S \wedge f(v) = k
    \end{cases}
    \]

    Charakteristiky čiastočných farbení budeme ďalej v texte značiť $c$.
\end{defn}

Z algoritmu Junosza-Szaniawski môžeme vidieť, že na reprezentáciu množiny čiastočných farbení
s obmedzeným rozsahom nám stačí ukladať množinu charakteristík fixného rádu. Keď nepotrebujeme
pre každý ofarbený vrchol ukladať jeho presnú farbu, môžeme si dovoliť reprezentovať aj
vrcholové separátory a použiť ich na rozdelenie problému. O presnom spôsobe a aplikácií
takéhoto rozdelenia si povieme v ďalšej časti.

Predtým si ale stručne pripomenieme lemu o hranových separátoroch z predošlej kapitoly \ref{hrsep-lema}.
Táto lema hovorí, že existencia $L(2,1)$-farbenia grafu $G$ vyplýva z existencie $L(2,1)$-farbení
komponentov grafu $G-E$ pre hranový oddeľovač $E$. Tieto farbenia musia navyše farbiť všetky vrcholy
z hranového oddeľovača a musia sa zhodovať v ich farbách.

Podobné tvrdenie platí aj pre čiastočné farbenia. V tomto prípade je podmienkou, že všetky čiastočné
farbenia farbia tú istú podmnožinu vrcholov z hranového oddeľovača a navyše im priradzujú rovnakú farbu.
Keďže ide o takmer identické tvrdenie k leme o hranových separátoroch \ref{hrsep-lema}, nebudeme
jeho presné znenie uvádzať.

\section{Rozdelenie na vrcholových separátoroch}

Základnou myšlienkou rozdelenia na vrcholových separátoroch je, že ak máme zafixované čiastočné
farbenie separátora a všetkých jeho susedov, farbenie zvyšných vrcholov v rôznych komponentoch
je nezávislé.

\begin{lema}
    Nech $S_V$ je vrcholový separátor grafu $G$, nech $G_1, G_2, \ldots G_k$ sú komponenty
    grafu $G - S_V$ a nech $H_1, H_2, \ldots, H_k$ sú grafy indukované množinami $V(G_1) \cup S_V,\ V(G_2) \cup S_V,\ \ldots,\ V(G_k) \cup S_V$.
    Nech $f$ je čiastočné $L(2,1)$-farbenie nejakej podmnožiny $S \subseteq N[S_V]$ v grafe $G$ a nech $f_1, f_2, \ldots f_k$
    sú čiastočné farbenia množín $S_1, S_2, \ldots, S_k$ v grafoch $H_1, H_2, \ldots, H_k$ (v tomto poradí), ktoré spĺňajú podmienku

    $$ \forall i \in \{1 \ldots k\},\ \forall v \in V(H_i) \cap N[S_V]: (v \in S_i \Leftrightarrow v \in S) \wedge (v \in S_i \Rightarrow f_i(v) = f(v)).$$

    Potom zobrazenie $\omega: \bigcup \limits_{i=1}^k S_i \to \mathbb{N}$ definované nasledovne:

    \[ \omega(v) =
    \begin{cases}
        f(v), & v \in S \\
        f_i(v), & v \in S_i - S
    \end{cases}
    \]

    tvorí čiastočné farbenie grafu $G$.
\end{lema}

\begin{proof}
    Dôkaz je veľmi podobný dôkazu lemy o hranových separátoroch \ref{hrsep-lema} v predošlej kapitole. Všetky
    vrcholy ofarbené vrámci jedného grafu $H_i$ nemôžu porušiť podmienku čiastočného $L(2,1)$-farbenia. Podobne
    pre dva vrcholy v $S$.
    
    Každá dvojica vrcholov, kde jeden patrí do $S_i - S$ a druhý do $S - S_i$ má vzdialenosť
    aspoň $3$, lebo cesta medzi nimi musí obsahovať aspoň jeden úsek, ktorý tvorí hrana z $S_i - S$ do $N(S)$, hrana
    z $N(S)$ do $S$ a hrana z $S$ do iného vrchola $N(S)$.

    Nakoniec každá dvojica vrcholov, kde jeden patrí do $S_i - S$ a druhý do $S_j - S$, kde $i \neq j$, musí mať tiež
    vzdialenosť aspoň $3$.
\end{proof}

Okolie jedného vrchola môže byť pomerne veľké, napríklad vo hviezdach v ňom môžu byť všetky zvyšné vrcholy.
Pri reprezentácii čiastočného farbenia cez jeho charakteristiku nám to však nevadí. Rôznych charakteristík,
ktoré môžu mať čiastočné farbenia malého separátora a jeho okolia, je malý. Horný odhad pre počet rôznych
charakteristík si zhrnieme v nasledujúcej leme.

\begin{lema}
    Nech $S$ je vrcholový separátor grafu $G$, nech $q = |S|$ a $r = |N[S]|$, nech $k \in \mathbb{N}$ je ľubovoľné
    číslo, nech $F$ je množina všetkých čiastočných farbení s rozsahom $k$, ktoré ofarbujú nejakú podmnožinu $N[S]$,
    nech $C$ je množina všetkých charakteristík rádu $k$ farbení v $F$. Potom $|S| \leq 2^r \cdot (r+1)^{q}$.
\end{lema}
\begin{proof}
    Najprv zhora odhadneme počet rôznych množín vrcholov, ktoré majú priradenú farbu $k$. Pre každý vrchol
    $v \in S$ platí, že v jeho okolí $N[v]$ je nanajvýš jeden vrchol, ktorý má farbu $k$. V opačnom prípade by totiž
    mali dva vrcholy vo vzdialenosti nanajvýš $2$ priradenú tú istú farbu. Keďže celkový počet vrcholov
    v $N[S]$ je $r$, pre každý z $q$ vrcholov separátora môžeme vybrať nanajvýš jednu z $r$ možností, alebo
    nevybrať žiadny vrchol. Preto množín vrcholov, ktoré majú priradenú farbu $k$, je nanajvýš $(r+1)^q$.

    Ďalej pre každú množinu vrcholou s farbou $k$ ostáva nanajvýš $r$ vrcholov, každý z nich nezávisle
    patrí, alebo nepatrí do množiny ofarbených vrcholov. Pre každú možnosť množiny vrcholov s farbou $k$
    máme teda nanajvýš $2^r$ možností, ako vyzerá množina ostatných ofarbených vrcholov. Charakteristík
    je teda nanajvýš $2^r (r+1)^q$.
\end{proof}

Rozdeľovanie pomocou vrcholového separátora budeme využívať na konštrukciu rýchlejších algoritmov
pre problém $L(2,1)$-farbenia. Pre všetky možné charakteristiky separátora a okolia si budeme pamätať,
aké charakteristiky môžu mať farbenia vo všetkých komponentoch.

Podobne bude prebiehať pridanie novej
farby. Vyskúšame pridanie všetkých farieb do separátora a jeho okolia. Pre každú takúto možnosť potom
pridáme novú farbu do každého komponentu. Ako to bude presne vyzerať s postupom a s časovou zložitosťou,
si ukážeme na príklade planárnych grafov.

\section{Farbenie planárnych grafov}

Dôležitými výsledkami pre farbenie planárnych grafov sú vety o separátoroch v planárnych grafoch \cite{tarjan_plansep},
ktoré si uvedieme.

\begin{veta}
    \label{planarsep-veta}
    Nech $G$ je planárny graf s $n$ vrcholmi. Potom sa dajú vrcholy grafu $G$ rozdeliť do troch množín $A, B, C$ tak,
    že neexistuje hrana medzi žiadnym vrcholom v $A$ a žiadnym vrcholom v $B$, množiny $A$ aj $B$ majú nanajvýš $\frac{n}{2}$
    vrcholov a množina $C$ má nanajvýš $\frac{2 \sqrt{2n}}{1 - \sqrt{\frac{2}{3}}}$ vrcholov.
\end{veta}

\begin{pozn}
Hodnota výrazu $\frac{2 \sqrt{2}}{1 - \sqrt{\frac{2}{3}}}$ je menšia ako $16$, preto budeme kvôli prehľadnosti
veľkosť množiny $C$ zhora odhadovať ako $16 \sqrt{n}$.
\end{pozn}

Ďalej budeme vytvárať algoritmus, ktorý dokáže pre danú množinu charakteristík čiastočných $L(2,1)$-farbení
s rozsahom $k$ vypočítať množinu charakteristík čiastočných $L(2,1)$-farbení s rozsahom $k+1$. Túto funkciu
nad množinami charakteristík budeme ďalej označovať $\oplus$.
Pre vysvetlenie nášho spôsobu počítania $\oplus(C)$ budeme potrebovať jednu ďalšiu definíciu.

\begin{defn}
    Nech $C$ je množina charakteristík čiastočných $L(2,1)$-farbení grafu $G$, nech $S = (s_1, s_2, \ldots, s_k)$ je nejaká postupnosť
    rôznych vrcholov v $G$ dlhá $k$ a nech $O = (o_1, o_2, \ldots, o_k)$ je postupnosť dlhá $k$ zložená zo znakov $0, 1, \bar{1}$.
    Potom pod symbolickým zápisom $C_{S=O}$ budeme rozumieť nasledujúcu množinu charakteristík:
    
    $$C_{S=O} = \{c \in C | \forall i \in \{1 \ldots k\}\ c(s_i) = o_i\}.$$
\end{defn}


FIXME:
Najprv rozdelíme množinu $C$ na (disjunktné) podmnožiny podľa toho, aké hodnoty majú priradené vrcholy
separátora $S$ a jeho okolia $N[S]$. 
Formálny zápis výpočtu bude vyzerať nasledovne:

$$ \oplus(C) = \oplus(\bigcup \limits_{}^{})$$

Riešenie
problému $L(2,1)$ farbenia potom niekoľkokrát aplikuje tento algoritmus, až kým nebude v množine
charakteristík $C$ taká, ktorá zodpovedá $L(2,1)$-farbeniu. Keďže každý graf má triviálne farbenie
s rozsahom $2n - 2$, tento algoritmus nám zaručene stačí spustiť najviac toľkokrát.

Základom algoritmu bude nájsť separátor podľa vety o planárnom separátore \ref{planarsep-veta}. Tento
krok stačí vykonať iba raz, preto si môžeme dovoliť jeho hľadanie tak, že prejdeme všetkých $2^n$ podmnožín
vrcholov a pre každú v polynomiálnom čase overíme, či je dostatočne malá a či po jej odstránení bude mať
každý komponent súvislosti nanajvýš $\frac{n}{2}$ vrcholov.

Ďalej máme k dispozícii vrcholový separátor $S$. Podľa toho, koľko vrcholov je v okolí separátora, čiže
v $N[S]$ budeme robiť rôzne veci. Najprv rozoberieme jednoduchší prípad, kde $|N[S]| \ge \frac{3n}{4}$,
potom rozoberieme prípad $|N[S]| < \frac{3n}{4}$.


