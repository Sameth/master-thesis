\chapter{FIXME: Vymysliet nazov}

FIXME: uvod

\begin{defn}
    Nech $f$ je čiastočné $L(2,1)$-farbenie grafu $G$ s rozsahom $k$. Pod pojmom \emph{charakteristika rádu $k$
    zobrazenia $f$} budeme rozumieť dvojicu $(F_P, F_O)$, kde $F_P$ je množina vrcholov ofarbená
    hodnotou $k$ a $F_O$ je množina ostatných ofarbených vrcholov.
\end{defn}



\section{Rozdelenie na vrcholových separátoroch}

\begin{lema}
    Nech $S_V$ je vrcholový separátor grafu $G$, nech $G_1, G_2, \ldots G_k$ sú komponenty
    grafu $G - S_V$ a nech $H_1, H_2, \ldots, H_k$ sú grafy indukované množinami $V(G_1) \cup S_V,\ V(G_2) \cup S_V,\ \ldots,\ V(G_k) \cup S_V$.
    Nech $f$ je čiastočné $L(2,1)$-farbenie nejakej podmnožiny $S \subseteq N[S_V]$ v grafe $G$ a nech $f_1, f_2, \ldots f_k$
    sú čiastočné farbenia množín $S_1, S_2, \ldots, S_k$ v grafoch $H_1, H_2, \ldots, H_k$ (v tomto poradí), ktoré spĺňajú podmienku

    $$ \forall i \in \{1 \ldots k\},\ \forall v \in V(H_i) \cap N[S_V]: (v \in S_i \Leftrightarrow v \in S) \wedge (v \in S_i \Rightarrow f_i(v) = f(v)).$$

    Potom zobrazenie $\omega: \bigcup \limits_{i=1}^k S_i \to \mathbb{N}$ definované nasledovne:

    \[ \omega(v) =
    \begin{cases}
        f(v), & v \in S \\
        f_i(v), & v \in S_i - S
    \end{cases}
    \]

    tvorí čiastočné farbenie grafu $G$.
\end{lema}

\begin{proof}
    Dôkaz je veľmi podobný dôkazu lemy o hranových separátoroch \ref{hrsep-lema} v predošlej kapitole. Všetky
    vrcholy ofarbené vrámci jedného grafu $H_i$ nemôžu porušiť podmienku čiastočného $L(2,1)$-farbenia. Podobne
    pre dva vrcholy v $S$.
    
    Každá dvojica vrcholov, kde jeden patrí do $S_i - S$ a druhý do $S - S_i$ má vzdialenosť
    aspoň $3$, lebo cesta medzi nimi musí obsahovať aspoň jeden úsek, ktorý tvorí hrana z $S_i - S$ do $N(S)$, hrana
    z $N(S)$ do $S$ a hrana z $S$ do iného vrchola $N(S)$.

    Nakoniec každá dvojica vrcholov, kde jeden patrí do $S_i - S$ a druhý do $S_j - S$, kde $i \neq j$, musí mať tiež
    vzdialenosť aspoň $3$.
\end{proof}
