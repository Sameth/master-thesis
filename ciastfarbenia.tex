\chapter{FIXME: Vymysliet nazov}

Pri všetkých významnejších algoritmoch na hľadanie farbiaceho čísla všeobecného grafu $G$
sa uplatnila kompaktnejšia reprezentácia množiny čiastočných farbení. V tejto kapitole sa pozrieme na to,
ako sa dá kompaktná reprezentácia použiť pri delení problému a ukážeme si algoritmy na
rozdeliteľných grafoch, ktoré budú dosahovať lepšiu časovú zložitosť, ako algoritmus
Junosza-Szaniawski.

Ako prvú si uvedieme definíciu našej kompaktnej reprezentácie čiastočných ofarbení.

\begin{defn}
    Nech $G$ je graf s vrcholmi usporiadanými v postupnosti $v_1,v_2, \ldots, v_n$.
    Nech $f$ je čiastočné $L(2,1)$-farbenie množiny $S$ v grafe $G$ s rozsahom $k$. Pod pojmom \emph{charakteristika rádu $k$
    zobrazenia $f$} budeme rozumieť reťazec $r \in \{0, 1, \bar{1}\}^n$, ktorý spĺňa:
    \begin{align*}
        r_i = 0 & \Leftrightarrow v_i \notin S \\
        r_i = 1 & \Leftrightarrow v_i \in S \wedge f(v_i) \neq k \\
        r_i = \bar{1} & \Leftrightarrow v_i \in S \wedge f(v_i) = k
    \end{align*}

    Ďalej pod pojmom \emph{rozšírená charakteristika rádu $k$ zobrazenia $f$} budeme rozumieť reťazec
    $s \in \{0, \bar{0}, 1, \bar{1}\}^n$, ktorý spĺňa
    \begin{align*}
        s_i = 0 & \Leftrightarrow v_i \notin S \wedge (\forall j \in \{1 \dots n\}: v_j \in N(v_i) \Rightarrow t_j \neq 1 \\
        s_i = \bar{0} & \Leftrightarrow v_i \notin S \wedge (\exists j \in \{1 \dots n\}: v_j \in N(v_i) \wedge t_j = 1 \\
        s_i = 1 & \Leftrightarrow v_i \in S \wedge f(v_i) \neq k \\
        s_i = \bar{1} & \Leftrightarrow v_i \in S \wedge f(v_i) = k
    \end{align*}

    Charakteristiky čiastočných farbení budeme ďalej v texte označovať symbolom $r$.
\end{defn}

Podobne vieme zadefinovať charakteristiku (resp. rozšírenú charakteristiku) množiny $C$ vo
farbení $f$ ako podpostupnosť charakteristiky (resp. rozšírenej charakteristiky) zobrazenia
$f$, ktorá obsahuje iba znaky zodpovedajúce vrcholom v $C$.

Z algoritmu Junosza-Szaniawski môžeme vidieť, že na reprezentáciu množiny čiastočných farbení
s obmedzeným rozsahom nám stačí ukladať množinu charakteristík fixného rádu. Keď nepotrebujeme
pre každý ofarbený vrchol ukladať jeho presnú farbu, môžeme si dovoliť reprezentovať aj
vrcholové separátory a použiť ich na rozdelenie problému. O presnom spôsobe a aplikácií
takéhoto rozdelenia si povieme v ďalšej časti.

FIXME: Pridať nejakú poznámku o tom, ako môžu charakteristiky vyzerať (dve $\bar{1}$ musia byť
dostatočne ďaleko).

Predtým si ale stručne pripomenieme lemu o hranových separátoroch z predošlej kapitoly \ref{hrsep-lema}.
Táto lema hovorí, že existencia $L(2,1)$-farbenia grafu $G$ vyplýva z existencie $L(2,1)$-farbení
komponentov grafu $G-E$ pre hranový oddeľovač $E$. Tieto farbenia musia navyše farbiť všetky vrcholy
z hranového oddeľovača a musia sa zhodovať v ich farbách.

Podobné tvrdenie platí aj pre čiastočné farbenia. V tomto prípade je podmienkou, že všetky čiastočné
farbenia farbia tú istú podmnožinu vrcholov z hranového oddeľovača a navyše im priradzujú rovnakú farbu.
Keďže ide o takmer identické tvrdenie k leme o hranových separátoroch \ref{hrsep-lema}, nebudeme
jeho presné znenie uvádzať.

\section{Rozdelenie na vrcholových separátoroch}

Základnou myšlienkou rozdelenia na vrcholových separátoroch je, že ak máme zafixované čiastočné
farbenie separátora a všetkých jeho susedov, farbenie zvyšných vrcholov v rôznych komponentoch
je nezávislé.

\begin{lema}
    Nech $S_V$ je vrcholový separátor grafu $G$, nech $G_1, G_2, \ldots G_k$ sú komponenty
    grafu $G - S_V$ a nech $H_1, H_2, \ldots, H_k$ sú grafy indukované množinami $V(G_1) \cup S_V,\ V(G_2) \cup S_V,\ \ldots,\ V(G_k) \cup S_V$.
    Nech $f$ je čiastočné $L(2,1)$-farbenie nejakej podmnožiny $S \subseteq N[S_V]$ v grafe $G$ a nech $f_1, f_2, \ldots f_k$
    sú čiastočné farbenia množín $S_1, S_2, \ldots, S_k$ v grafoch $H_1, H_2, \ldots, H_k$ (v tomto poradí), ktoré spĺňajú podmienku
    $$ \forall i \in \{1 \ldots k\},\ \forall v \in V(H_i) \cap N[S_V]: (v \in S_i \Leftrightarrow v \in S) \wedge (v \in S_i \Rightarrow f_i(v) = f(v)).$$

    Potom zobrazenie $\omega: \bigcup \limits_{i=1}^k S_i \to \mathbb{N}$ definované nasledovne:

    \[ \omega(v) =
    \begin{cases}
        f(v), & v \in S \\
        f_i(v), & v \in S_i - S
    \end{cases}
    \]

    tvorí čiastočné farbenie grafu $G$.
\end{lema}

\begin{proof}
    Dôkaz je veľmi podobný dôkazu lemy o hranových separátoroch \ref{hrsep-lema} v predošlej kapitole. Všetky
    vrcholy ofarbené vrámci jedného grafu $H_i$ nemôžu porušiť podmienku čiastočného $L(2,1)$-farbenia. Podobne
    pre dva vrcholy v $S$.
    
    Každá dvojica vrcholov, kde jeden patrí do $S_i - S$ a druhý do $S - S_i$ má vzdialenosť
    aspoň $3$, lebo cesta medzi nimi musí obsahovať aspoň jeden úsek, ktorý tvorí hrana z $S_i - S$ do $N(S)$, hrana
    z $N(S)$ do $S$ a hrana z $S$ do iného vrchola $N(S)$.

    Nakoniec každá dvojica vrcholov, kde jeden patrí do $S_i - S$ a druhý do $S_j - S$, kde $i \neq j$, musí mať tiež
    vzdialenosť aspoň $3$.
\end{proof}

Okolie jedného vrchola môže byť pomerne veľké, napríklad vo hviezdach v ňom môžu byť všetky zvyšné vrcholy.
Pri reprezentácii čiastočného farbenia cez jeho charakteristiku nám to však nevadí. Rôznych charakteristík,
ktoré môžu mať čiastočné farbenia malého separátora a jeho okolia, je malý. Horný odhad pre počet rôznych
charakteristík si zhrnieme v nasledujúcej leme.

\begin{lema}
    Nech $S$ je vrcholový separátor grafu $G$, nech $q = |S|$ a $p = |N[S]|$, nech $k \in \mathbb{N}$ je ľubovoľné
    číslo, nech $F$ je množina všetkých čiastočných farbení s rozsahom $k$, ktoré ofarbujú nejakú podmnožinu $N[S]$,
    nech $R$ je množina všetkých charakteristík rádu $k$ farbení v $F$. Potom $|R| \leq 2^p \cdot (p+1)^{q}$.
\end{lema}
\begin{proof}
    Najprv zhora odhadneme počet rôznych množín vrcholov, ktoré majú priradenú farbu $k$. Pre každý vrchol
    $v \in S$ platí, že v jeho okolí $N[v]$ je nanajvýš jeden vrchol, ktorý má farbu $k$. V opačnom prípade by totiž
    mali dva vrcholy vo vzdialenosti nanajvýš $2$ priradenú tú istú farbu. Keďže celkový počet vrcholov
    v $N[S]$ je $p$, pre každý z $q$ vrcholov separátora môžeme vybrať nanajvýš jednu z $p$ možností, alebo
    nevybrať žiadny vrchol. Preto množín vrcholov, ktoré majú priradenú farbu $k$, je nanajvýš $(p+1)^q$.

    Ďalej pre každú množinu vrcholou s farbou $k$ ostáva nanajvýš $p$ vrcholov, každý z nich nezávisle
    patrí, alebo nepatrí do množiny ofarbených vrcholov. Pre každú možnosť množiny vrcholov s farbou $k$
    máme teda nanajvýš $2^p$ možností, ako vyzerá množina ostatných ofarbených vrcholov. Charakteristík
    je teda nanajvýš $2^p (p+1)^q$.
\end{proof}

Rozdeľovanie pomocou vrcholového separátora budeme využívať na konštrukciu rýchlejších algoritmov
pre problém $L(2,1)$-farbenia. Pre všetky možné charakteristiky separátora a okolia si budeme pamätať,
aké charakteristiky môžu mať farbenia vo všetkých komponentoch.

Podobne bude prebiehať pridanie novej
farby. Vyskúšame pridanie všetkých farieb do separátora a jeho okolia. Pre každú takúto možnosť potom
pridáme novú farbu do každého komponentu. Ako to bude presne vyzerať s postupom a s časovou zložitosťou,
si ukážeme na príklade planárnych grafov.

\section{Farbenie planárnych grafov}

Dôležitým výsledkom pre farbenie planárnych grafov je veta o separátoroch v planárnych grafoch \cite{tarjan_plansep},
ktorú si uvedieme.

\begin{veta}
    \label{planarsep-veta}
    Nech $G$ je planárny graf s $n$ vrcholmi. Potom sa dajú vrcholy grafu $G$ rozdeliť do troch množín $A, B, C$ tak,
    že neexistuje hrana medzi žiadnym vrcholom v $A$ a žiadnym vrcholom v $B$, množiny $A$ aj $B$ majú nanajvýš $\frac{n}{2}$
    vrcholov a množina $C$ má nanajvýš $\frac{2 \sqrt{2n}}{1 - \sqrt{\frac{2}{3}}}$ vrcholov.
\end{veta}

\begin{pozn}
Hodnota výrazu $\frac{2 \sqrt{2}}{1 - \sqrt{\frac{2}{3}}}$ je menšia ako $16$, preto budeme kvôli prehľadnosti
veľkosť množiny $C$ zhora odhadovať ako $16 \sqrt{n}$.
\end{pozn}

Ďalej budeme vytvárať algoritmus, ktorý dokáže pre danú množinu charakteristík čiastočných $L(2,1)$-farbení
s rozsahom $k$ vypočítať množinu charakteristík čiastočných $L(2,1)$-farbení s rozsahom $k+1$. Túto funkciu
nad množinami charakteristík budeme ďalej označovať $\oplus$. Pre množinu charakteristík $R$ zadefinujeme
množinu suffixov reťazca $w$ ako $R_w = \{u | wu \in R \}$. Budeme potrebovať ešte jeden pojem:

\begin{defn}
    Nech $G$ je graf. Dvojicu charakteristík $(r_1, r_2)$ budeme nazývať \emph{prechod rádu $k$}, ak existujú
    čiastočné $L(2,1)$-farbenia $f_1, f_2$, ktoré spĺňajú nasledujúce podmienky:
    \begin{enumerate}
        \item $r_1$ je charakteristika $f_1$ a $f_2$ je charakteristika $r_2$,
        \item $\forall i \leq k : \{v \ |\ f_1(v) = i\} = \{v \ |\ f_2(v) = i\}$.
    \end{enumerate}

    Dvojicu charakteristík $(s_1, s_2)$ množiny $S \subseteq V(G)$ budeme nazývať \emph{prechod rádu $k$ nad množinou $S$}, ak
    sa $s_1$ a $s_2$ dajú doplniť na prechod grafu $G$.

    Počet všetkých prechodov v $G$ označíme $pr(G)$. Podobne počet prechodov nad množinou $S$ v grafe $G$
    označíme $pr_S(G)$.
\end{defn}

Vrcholy separátora a okolia (množinu $N[C]$, kde $C$ je množina z vety \ref{planarsep-veta})
označíme $S$, počet vrcholov v $S$ označíme $p$. Vrcholy grafu usporiadame tak, aby vrcholy $S$ boli na začiatku.

Základný postup pri počítaní $\oplus(R)$ s využitím separátorov bude nasledovný:
Množinu (nerozšírených) charakteristík $R$ rozdelíme podľa ohodnotenia $S$. Pre každý prechod
nad $S$ nezávisle vypočítame $\oplus$ na charakteristikách zvyšných vrcholov. Nakoniec tieto
čiastočné výsledky pospájame a odstránime duplikáty.
Formálny zápis výpočtu a jeho odvodenia bude vyzerať nasledovne:

$$ \oplus(R) = \bigcup \limits_{w \in R} \oplus(w) = \bigcup \limits_{w \in \{0, 1, \bar{1}\}^p} \oplus(R_w) = \bigcup \limits_{u \in \{0, 1, \bar{1}\}^p} u \cdot \left(\bigcup \limits_{\substack{w \in \{0, 1, \bar{1}\}^p\\ \textrm{t.ž. } u \in \oplus(w)}} \oplus(R_w) \right)$$

Riešenie
problému $L(2,1)$ farbenia potom niekoľkokrát aplikuje tento algoritmus, až kým nebude v množine
charakteristík $R$ taká, ktorá zodpovedá $L(2,1)$-farbeniu. Keďže každý graf má triviálne farbenie
s rozsahom $2n - 2$, tento algoritmus nám zaručene stačí spustiť najviac $(2n - 2)$-krát.

Základom algoritmu bude nájsť separátor podľa vety o planárnom separátore \ref{planarsep-veta}. Tento
krok stačí vykonať iba raz, preto si môžeme dovoliť jeho hľadanie tak, že prejdeme všetkých $2^n$ podmnožín
vrcholov a pre každú v polynomiálnom čase overíme, či je dostatočne malá a či po jej odstránení bude mať
každý komponent súvislosti nanajvýš $\frac{n}{2}$ vrcholov.

Ďalej teda predpokladáme, že poznáme vrcholový separátor $C$. Podľa toho, koľko vrcholov je v okolí separátora, čiže
v $N[C]$, budeme robiť rôzne veci. Najprv rozoberieme jednoduchší prípad, kde $|N[C]| \ge \frac{3n}{4}$,
potom rozoberieme prípad $|N[C]| < \frac{3n}{4}$.

\subsection{Separátor s veľkým okolím}

Popíšeme algoritmus, ktorý pre danú množinu charakteristík $R$ rádu $k$ spočíta novú množinu charakteristík
rádu $k+1$. Základný postup sme už popísali, pre každú možnosť starej a novej charakteristiky separátora
vyskúšame pridať novú farbu zvyšným vrcholom. Separátor označíme $S$, počet vrcholov v $N[S]$ označíme $p$, počet
vrcholov $S$ označíme $q$.

Pre ľubovoľnú charakteristiku separátora a jeho okolia existuje najviac $(p+1)^q$ možností, ktoré
vrcholy dostanú novú farbu $k+1$. Taktiež vieme všetky tieto možnosti generovať -- pre každý vrchol v separátore
vyberieme niektorý z vrcholov jeho uzavretého okolia, alebo nevyberieme v jeho okolí žiadny vrchol. Pre
každý takýto výber následne overíme, či sme nevybrali dva vrcholy vo vzdialenosti menšej, ako $3$, alebo
či nejaký vybratý vrchol nesusedí s vrcholom farby $k$.

Pre každú možnosť starej a novej charakteristiky separátora ostane nejaka množina charakteristík
zvyšných $n-p \leq \frac{n}{4}$ vrcholov, pre ktorú potrebujeme vypočítať nové charakteristiky. Tu si môžeme
dovoliť extrémne neefektívne riešenie, kde pre každú starú charakteristiku $r$ a každú podmnožinu $U$
jej neofarbených vrcholov vyskúšame, či je povolené ofarbiť všetky vrcholy $U$ farbou $k+1$. Takýmto
spôsobom budeme overovať nanajvýš $4^{n-p}$ možností -- keby boli vrcholy mimo separátora nezávislé,
pre každý máme iba $4$ typy možností, v ktorých ho budeme skúšať: Neofarbený vrchol ostáva neofarbený,
neofarbený vrchol farbíme $k+1$, vrchol ofarbený farbou $k$, vrchol ofarbený inou farbou.

Overenie
jedného výberu je polynomiálne, lebo stačí pre každú dvojicu vrcholov vo vzdialenosti nanajvýš
$2$ skontrolovať, či by ich priradenie farieb neporušilo podmienky $L(2,1)$-farbenia.

Pre každú z nanajvýš $2^p (p+1)^q$ možností ofarbenia $S$ vyskúšame nanajvýš $(p+1)^q$ možností pridania
novej farby do $S$. Pre každú takúto možnosť následne vyskúšame nanajvýš $4^{n-p}$ možností pridania novej
farby do množiny charakteristík a pre každú možnosť vykonáme polynomiálne veľa práce.

Celková časová
zložitosť teda bude $O^*(2^p (p+1)^{2q} 4^{n-p})$. Z vety o separátore v planárnych grafoch \ref{planarsep-veta}
vieme, že veľkosť separátora, $q$, je nanajvýš $16\sqrt{n}$. Túto vetvu algoritmu spúšťame, keď $p \ge \frac{3n}{4}$.
Funkcia $2^p 4^{n-p}$ klesá s $p$ a teda jeho najväčšia možná hodnota pre minimálne $p = \frac{3n}{4}$ je
$2^{\frac{5n}{4}}$. Výraz $(p+1)^{2q}$ zhora odhadneme na $(n+1)^{32\sqrt{n}} \leq (2n)^{32\sqrt{n}}$.

Dosadením horných odhadov dostaneme časovú zložitosť $O^*\left(2\strut^{\frac{5n}{4} + 32\sqrt{n} + 32\lg{n}\sqrt{n}}\right) = O^*\left(2.38\strut^{n + o(n)}\right)$.

\subsection{Separátor s malým okolím}

V prípade separátora s malým okolím budeme využívať rozdelenie pomocou separátora. Nech $A, B, C$ sú množiny
vrcholov spĺňajúce vetu o planárnom separátore \ref{planarsep-veta}. Bez ujmy na všeobecnosti má množina $C$
viac susedov v množine $A$, než v množine $B$. Keďže dokopy má množina $N[C]$ menej ako $\frac{3n}{4}$ vrcholov
a množiny $A$ a $B$ sú disjunktné, množina $B \cap N[C]$ má nanajvýš polovicu z tohto počtu, čiže $\frac{3n}{8}$.

Pre konštrukciu algoritmu použijeme hranový separátor medzi množinami $B$ a $C$, ktorý obsahuje vrcholy
$C \cup (B \cap N[C])$. Túto množinu vrcholov budeme ďalej označovať $S$. Množinu hrán medzi množinami
$B$ a $C$ označíme $E$. Komponenty súvislosti grafu $G-E$ označíme $K_1, \ldots, K_l$. Počet vrcholov v $S$
označíme $p$, počet vrcholov v $C$ označíme $q$. FIXME: ďalšie označenia

Z lemy o hranových separátoroch vieme, že pre zafixovanú charakteristiku vrcholov $S$ môžeme pre každý
z komponentov $K_1 \ldots K_l$ ukladať množinu jeho charakteristík nezávisle. Množinu charakteristík
celého grafu by sme vedeli zostrojiť ako karteziánsky súčin týchto dielčích charakteristík, zjednotený cez
všetky charakteristiky množiny $S$. Explicitne to však robiť nebudeme.

Postup počítania funkcie $\oplus$ bude podobný, ako pri separátore s veľkým okolím. Pre každý prechod nad $S$
a každý komponent $K_i$ skonštruujeme z jeho pôvodnej
množiny charakteristík $R_i$ novú množinu charakteristík $\oplus(R_i)$. Keďže každý komponent tvorí súvislý
graf, môžeme na počítanie $\oplus(R_i)$ použiť postup z algoritmu Junosza-Szaniawski.

Pre počítanie operácie $\oplus$ podľa algoritmu Junosza-Szaniawski potrebujeme poznať rozšírené charakteristiky.
Pre každý komponent a každú jeho (nerozšírenú) charakteristiku vieme vypočítať jej rozšírený ekvivalent v
polynomiálnom čase: Pre každý vrchol s hodnotou $\bar{1}$ prečíslujeme všetkých susedov s hodnotou $0$
na hodnotu $\bar{0}$.

Pri počítaní operácie $\oplus$ na komponentoch $K_1, \ldots, K_l$ neberieme do úvahy celý separátor,
vždy používame iba tú časť, ktorá sa nachádza v danom komponente. Môže teda nastať situácia, že
prechod na celom separátore nebude konzistentný s niektorými starými, alebo s niektorými novými
charakteristikami v komponente.

Toto vieme vyriešiť ľahko. Pre
každý prechod separátora $S$ budeme pred vykonaním $\oplus$ na komponente $K_i$ filtrovať jeho
charakteristiky, aby sme používali iba tie, pri ktorých môže daný prechod separátora nastať. Po vypočítaní
$\oplus$ na komponente $K_i$ môžeme taktiež dostať nejaké charakteristiky, ktoré nie sú kompatibilné
s prechodom separátora. Aj v tomto prípade stačí tieto charakteristiky odfiltrovať po vypočítaní
operácie $\oplus$.

V krátkosti si popíšeme, ako bude fungovať filtrovanie pred počítaním operácie $\oplus$ v komponente $K_i$.
Filtrovanie po výpočte $\oplus$ sa dá zostrojiť analogicky.

Zistiť, či je charakteristika kompatibilná s prechodom separátora, vieme v polynomiálnom čase. Pre
každý vrchol $v$ v komponente $K_i$ a pre každý vrchol $u$ separátora skontrolujeme, či sú vzdialené
aspoň $3$. Ak nie sú, môžu byť vzdialené $2$ alebo $1$. Ak sú vzdialené $2$, nesmie nastať situácia,
že $v$ má charakteristiku $\bar{1}$ a $u$ má starú charakteristiku $\bar{1}$. Ak sú vzdialené $1$,
nesmie nastať situácia, že $v$ má charakteristiku $\bar{1}$ a $u$ má starú, alebo novú charakteristiku
$\bar{1}$.

\subsubsection{Časová zložitosť}

Všetky komponenty, ktoré obsahujú nejaký vrchol v $B$, majú nanajvýš $\frac{n}{2}$ vrcholov, lebo
oddeľovač $E$ oddeľuje množinu $B$ a $C$. Množiny $A$ a $B$ sú oddelené podľa vety o separátore
a množina $B$ má nanajvýš $\frac{n}{2}$ vrcholov.

Podobne, všetky komponenty, ktoré obsahujú nejaký vrchol v $A$, majú nanajvýš $\frac{n}{2} + 16 \sqrt{n}$
vrcholov, lebo množiny $A$ a $C$ majú dohromady najviac $\frac{n}{2} + 16 \sqrt{n}$ vrcholov a ich vrcholy
nie sú spojené s vrcholmi v $B$.

Počet charakteristík $S$ odhadneme podobne, ako pre vrcholové separátory: Pre každý vrchol v $C$ máme
nanajvýš $p+1$ možností, ako vybrať nanajvýš jeden vrchol z jeho okolia. Výberov pre všetky vrcholy je
teda nanajvýš $(p+1)^q$ a pre zvyšné vrcholy máme nanajvýš $2^p$ možností. Dohromady dostávame horný
odhad $2^p (p+1)^q$ možností. Pre každú z týchto možností potom máme nanajvýš $(p+1)^q$ možností, ktoré
vrcholy dostanú farbu $k+1$.

Nakoniec pre každú možnosť starej charakteristiky a množiny vrcholov s novou farbou budeme počítať
na každom komponente operáciu $\oplus$ pomocou algoritmu z článku Junosza-Szaniawskeho a kol. \cite{junosza_fast},
ktorá na komponente s veľkosťou $x$ pracuje v čase $O^*(2.65^x)$. Každý komponent, na ktorom ju budeme
spúšťať, má nanajvýš $\frac{n}{2} + 16\sqrt{n}$ vrcholov. Komponentov je nanajvýš $n$, teda budeme
potrebovať nanajvýš $n$ nezávislých výpočtov. Tento faktor sa stratí v $O^*$ notácii.

Časová zložitosť tohto algoritmu teda bude $O^*\left(2\strut^{\frac{3n}{8}} \cdot (\frac{3n}{8} + 16 \sqrt{n} + 1)\strut^{32 \sqrt{n}} \cdot 2.65\strut^{\frac{n}{2} + 16 \sqrt{n}}\right) = O^*\left(2.12\strut^{n + o(n)}\right)$.

\subsection{Vylepšenie a poznámky}

Popísali sme si algoritmus, ktorý na základe veľkosti okolia separátora robil veľmi rôzne výpočty.
Pri vysvetľovaní sme ako hranicu použili ľahko zapísateľnú, ale nie najoptimálnejšiu konštantu
$\alpha = \frac{3}{4}$. Pre grafy, v ktorých má okolie separátora aspoň $\alpha n$ vrcholov
sme používali algoritmus pre separátor s veľkým okolím, pre ostatné algoritmus pre separátor
s malým okolím. Ďalej odhadneme lepšiu konštantu a časovú zložitosť, ktorá z jej použitia vyplýva.

Časová zložitosť algoritmu pre separátor s veľkým okolím je pre ľubovoľnú konštantu $\alpha \in (0, 1)$
v triede funkcií $O^*(2^{\alpha n + o(n)}4^{(1 - \alpha) n} = O^*(2^{n(2 - \alpha) + o(n)})$.

Časová zložitosť algoritmu pre separátor s malým okolím je pre ľubovoľnú konštantu $\alpha \in (0,1)$
v triede funkcií $O^*(2^{n\frac{\alpha}{2} + o(n)} 2.65^{\frac{n}{2} + o(n)})$.

Zvyšovaním konštanty $\alpha$ teda zlepšujeme časovú zložitosť algoritmu pre separátor s malým
okolím a zhoršujeme časovú zložitosť algoritmu pre separátor s veľkým okolím. Vyváženú časovú
zložitosť dostaneme pre takú konštantu $\alpha$, kde $2 - \alpha = \frac{\alpha}{2} + \frac{\lg(2.65)}{2}$.
Vyriešením tejto rovnice dostávame hodnotu $\alpha \approx 0.865$, z ktorej vyplýva časová zložitosť
$O^*(2.2^{n + o(n)})$ pre oba prípady.

Uvedený algoritmus sa okrem existencie vrcholového rozdeľovača veľkosti $O(\sqrt{n})$ nespoliehal
na žiadnu inú vlastnosť planárnych grafov. Preto uvedený algoritmus funguje pre všetky triedy grafov,
v ktorých existuje vrcholový separátor veľkosti $O(\sqrt{n})$. Toto tvrdenie vieme dokonca mierne
zosilniť na triedy grafov, v ktorých existuje vrcholový separátor veľkosti $O(n^{1 - \epsilon})$ pre
ľubovoľnú hodnotu $\epsilon > 0$.

FIXME: Napisat nejaky rozumny pseudokod. Ak ostane na taketo pompeznosti cas.

\section{Farbenie vyvážene rozdeliteľných grafov}

Na príkade planárnych grafov sme ukázali spôsob, ako využiť existenciu malého vrcholového
separátora pre konštrukciu rýchlejšieho algoritmu na hľadanie farbiaceho čísla grafu.
Už pri tejto konštrukcii sme potrebovali prejsť od vrcholového separátora k hranovému
separátoru, ktorý má menej vrcholov a teda aj rôznych charakteristík, resp. prechodov.
Ďalej túto myšlienku využijeme na konštrukciu algoritmu pre dobre rozdeliteľné grafy.

\begin{defn}
    Nech $G$ je jednoduchý graf s $n$ vrcholmi, nech $A, B$ sú podmnožiny $V(G)$,
    nech $C = N(B)$ a $D = N(A)$. Dvojicu množín $(A,B)$ budeme volať \emph{vyvážené rozdelenie
    grafu $G$}, ak platí:
    \begin{enumerate}
        \item $A \cup B = V(G) \wedge A \cap B = \emptyset$
        \item $|A| \leq \frac{2n}{3}$
        \item $|B| \leq \frac{2n}{3}$
        \item $|C \cup D| \leq \frac{n}{4}$
    \end{enumerate}

    Triedu grafov, v ktorých existuje vyvážené rozdelenie, budeme volať \emph{vyvážene rozdeliteľné}.
\end{defn}

Pre ľubovoľný graf vieme overiť, či je vyvážene rozdeliteľný, v čase $O^*(2^n)$ a v prípade, že je
vyvážene rozdeliteľný aj nájdeme jeho vyvážené rozdelenie. Pre každú z $2^n$ podmnožín vrcholov
v polynomiálnom čase overíme, či môže byť množinou vo vyváženom rozdelení.

Pre ľubovoľnú podmnožinu $X \subseteq V(G)$ naozaj vieme
v polynomiálnom čase overiť, či môže byť množinou vyváženého rozdelenia: Skontrolujeme, či má
nanajvýš $\frac{2n}{3}$ a aspoň $\lceil \frac{n}{3} \rceil$ vrcholov. Ďalej pre každý vrchol v $X$
overíme, či nejaký z jeho susedov leží mimo $X$ a počet takýchto vrcholov označíme $c$. Podobne
pre každý vrchol mimo $X$ overíme, či má suseda v $X$ a počet takýchto vrcholov označíme $d$. Ak
platí $c + d \leq \frac{n}{4}$, tak dvojica $(X, V(G) - X)$ tvorí vyvážené rozdelenie grafu $G$.

Algoritmus pre hľadanie farbiaceho čísla vyvážene rozdeliteľných grafov budeme konštruovať podobne, ako
algoritmus pre planárne grafy (s malým okolím separátora). Najprv si označíme objekty, s ktorými budeme
pracovať. Graf budeme tradične označovať $G$, jeho počet vrcholov $n$. Množiny vyváženého rozdelenia $G$
označíme $A$ a $B$. Množinu $N(B)$ označíme $C$ a množinu $N(A)$ označíme $D$. Do pozornosti dávame, že
platí $N(B) \subseteq A$ a $N(A) \subseteq B$. Separátor bude tvorený vrcholmi $C \cup D$ a zodpovedá hranovému
separátoru $\{(u,v) \in E(G) \ |\  u \in C \wedge v \in D\}$, ktorý označíme $E$. Komponenty grafu $G - E$ budeme
označovať $K_1 \ldots K_m$.

Základný postup bude nasledovný: Pre každú možnosť prechodu nad separátorom $C \cup D$
a každý komponent (nezávisle) vypočítame, aká
množina charakteristík v ňom môže byť po pridaní novej farby. Pre tento výpočet budeme používať
postup z algoritmu Junosza-Szaniawski.

Pre aplikovanie postupu z algoritmu Junosza-Szaniawski treba prerobiť charakteristiky na
rozšírené charakteristiky. Rovnako treba pred spustením a po spustení algoritmu pre daný
prechod separátora a daný komponent prefiltrovať množinu starých a nových charakteristík,
aby sme odstránili tie, ktoré nie sú kompatibilné s prechodom separátora. Tento problém
vyriešime rovnako, ako v prípade planárnych grafov, v ktorých má separátor malé okolie.

Ďalej budeme odhadovať časovú zložitosť tohto algoritmu. Najzložitejšia časť odhadu sa bude týkať
počtu rôznych prechodov v separátore, čiže počtu kombinácií starej a novej charakteristiky separátora.

\subsection{Horný odhad počtu prechodov}

Najprv dokážeme, že separátor vieme pokryť hviezdami. Potom odvodíme počet prechodov vo hviezde
s $h$ vrcholmi a nakoniec zhora ohraničíme počet prechodov v našom separátore $C \cup D$.

\begin{defn}
    Graf $G$ voláme \emph{hviezda}, ak v ňom existuje vrchol $v$, ktorý je spojený hranou s každým iným
    vrcholom, a každá hrana v $G$ je incidentná s vrcholom $v$. Vrchol $v$ spĺňajúci tieto podmienky
    budeme volať \emph{stred hviezdy}. Pre určený stred hviezdy budeme zvyšné vrcholy nazývať \emph{ramená}.

    Graf $G$ voláme \emph{netriviálna hviezda}, ak má aspoň dva vrcholy a je hviezda.
\end{defn}

\begin{lema}
    Nech $G$ je súvislý graf s aspoň dvomi vrcholmi. Potom existuje množina jeho podgrafov
    $H_1, H_2, \ldots H_k$, ktorá spĺňa nasledujúce podmienky:
    \begin{enumerate}
        \item $\forall i, j \in \{1 \ldots k\}, i \neq j: V(H_i) \cap V(H_j) = \emptyset$,
        \item $\forall v \in V(G) \exists i \in \{1 \ldots k\}: v \in V(H_i)$,
        \item $\forall i \in \{1 \ldots k\}: H_i \textrm{ je netriviálna hviezda}$.
    \end{enumerate}

    Množinu $H_1, H_2, \ldots, H_k$ budeme volať \emph{pokrytie grafu $G$ hviezdami}.
\end{lema}

\begin{proof}
    Dokážeme indukciou vzhľadom na počet vrcholov grafu, $n$.

    Báza indukcie, $n = 2$: Existuje iba jeden dvojvrcholový súvislý
    graf: obsahuje dva vrcholy prepojené hranou. Ľahko vidíme, že tento graf je hviezda a oba vrcholy
    môžu byť jej stredom.

    Indukčný krok, $n \to n+1$: V každom súvislom grafe $H$ existuje vrchol $u$ taký, že graf $H - u$ je súvislý.
    Nech $v$ je takýto vrchol v našom grafe $G$ a nech $H_1 \ldots H_k$ sú podgrafy $G-v$, ktoré spĺňajú
    podmienky lemy a ktorých existencia vyplýva z indukčného predpokladu. Keďže $G$ je súvislý graf s aspoň
    tromi vrcholmi, musí v ňom existovať aspoň jedna hrana $e$ medzi vrcholom $v$ a nejakým iným vrcholom $u$.
    Bez ujmy na všeobecnosti platí $u \in V(H_1)$.

    Ďalej rozoberieme niekoľko možností podľa toho, ako vyzerá hviezda $H_1$:

    \begin{description}
        \item[$H_1$ má dva vrcholy:] V tomto prípade môže byť stredom hviezdy ľubovoľný z vrcholov $H_1$.
        Preto graf $H_1 \cup \{e, v\}$ je hviezda s vrcholom $u$. Vyhovujúcou množinou pre lemu je teda
        $H_1 \cup \{e, v\}, H_2, \ldots, H_k$.
        \item[$H_1$ má aspoň tri vrcholy a $u$ je jeho stred:] Pokiaľ pripojíme k stredu hviezdy ďalší
        vrchol, dostaneme opäť hviezdu, preto $H_1 \cup \{e, v\}, H_2, \ldots H_k$ je množina hviezd
        spĺňajúca lemu.
        \item[$H_1$ má aspoň tri vrcholy a $u$ nie je jeho stred:] Keď z hviezdy odoberieme vrchol, ktorý
        nie je jej stredom, dostaneme opäť hviezdu. Keďže $H_1$ má aspoň tri vrcholy, $H_1 - u$ je netriviálna
        hviezda. Zároveň graf $H$ tvorený vrcholmi $u, v$ a hranou $e$ je netriviálna hviezda. Preto množina
        hviezd $H_1 - u, H_2, \ldots, H_k, H$ je množina hviezd spĺňajúca podmienky lemy.
    \end{description}
\end{proof}

Ďalej budeme odvádzať vzťah pre počet možných prechodov medzi charakteristikami hviezd v závislosti od veľkosti hviezdy.
Počet možných prechodov vieme zhora odhadnúť ako súčin medzi počtami pre jednotlivé hviezdy. Preto nás bude pre hviezdu
s $h$ vrcholmi zaujímať hlavne $h$-ta odmnocnina z jeho počtu prechodov. Uvidíme, že najvyššiu $h$-tu odmocninu budú
mať práve dvojvrcholové hviezdy. Najprv vypíšeme všetky prechody pre dvojvrcholové hviezdy a potom dokážeme všeobecný
vzťah pre hviezdy s viacerými vrcholmi.

\subsubsection{Prechody nad hviezdami}

Dvojvrcholová hviezda je tvorená dvomi vrcholmi, ktoré sú prepojené hranou. Charakteristiky budeme pre prehľadnosť
vypisovať ako dvojice čísel. Ešte neofarbený vrchol (s charakteristikou $0$), môže dostať novú farbu $k+1$
(charakteristika $\bar{1}$), ak jeho sused doteraz nemal farbu $k$ (charakteristika $\bar{1}$).
Sused, ktorý mal farbu $k$ (charakteristika $\bar{1}$) nebude mať najnovšiu farbu $k+1$, preto sa mu
zmení charakteristika na $1$.
\begin{align*}
    (0, 0) \to &\ (0, 0) \ | \ (0, \bar{1}) \ | \ (\bar{1}, 0) \\
    (0, 1) \to &\ (0, 1) \ | \ (\bar{1}, 1) \\
    (0, \bar{1}) \to &\ (0, 1) \\
    (1, 0) \to &\ (1, 0) \ | \ (1, \bar{1}) \\
    (1, 1) \to &\ (1, 1) \\
    (1, \bar{1}) \to &\ (1, 1) \\
    (\bar{1}, 0) \to &\ (1, 0) \\
    (\bar{1}, 1) \to &\ (1, 1)
\end{align*}

\begin{lema}
    Nech $H$ je netriviálna hviezda so stredom $u$ a s $x$ ramenami. Potom je počet prechodov medzi charakteristikami
    na $H$ presne $2^{x+1}\left(2 + x + \frac{x(x-1)}{4}\right)$.
\end{lema}
\begin{proof}
    Rozoberieme všetky možnosti. Pokiaľ má vrchol $u$ charakteristiku $\bar{1}$, žiadne z ramien nemohlo
    mať charakteristiku $\bar{1}$ a tiež nemohlo dostať novú farbu. Preto z každej charakteristiky, v ktorej
    má vrchol $u$ hodnotu $\bar{1}$, existuje len jeden prechod. Pre každé z $x$ ramien máme dve nezávislé
    možnosti ich hodnoty v charakteristike: $1$ a $0$. Preto je takýchto možností $2^x$.

    Pokiaľ má vrchol $u$ charakteristiku $1$, rozoberieme 4 prípady podľa toho, či má niektoré rameno
    hodnotu v starej charakteristike $\bar{1}$ a či má niektoré rameno hodnotu v novej charakteristike $\bar{1}$.
    Keďže vzdialenosť každej dvojice ramien je $2$, v žiadnej charakteristike nemôžu mať dve ramená priradenú
    zároveň hodnotu $\bar{1}$. Rozoberanie možností:

    \begin{description}
        \item[Stará aj nová charakteristika obsahuje rameno s hodnotou $\bar{1}$:] Spomedzi ramien môžeme
        nezávisle a usporiadane vybrať dve, jedno rameno bude mať v starej charakteristike hodnotu $\bar{1}$ a jedno rameno
        v novej charakteristike. Takýchto výberov je $x(x-1)$. Pre zvyšných $x-2$ ramien máme $2^{x-2}$ možností. $\leadsto x(x-1)2^{x-2}$
        \item[Stará charakteristika obsahuje rameno s hodnotou $\bar{1}$, nová nie:] Máme $x$ možností, ktoré
        rameno má charakteristiku $\bar{1}$. Pre zvyšných $x-1$ ramien máme $2^{x-1}$ možností. $\leadsto x2^{x-1}$
        \item[Stará charakteristika neobsahuje, nová obsahuje rameno s hodnotou $\bar{1}$:] Rovnako ako v
        predošlom prípade máme $x$ možností pre výber význačného ramena a $2^{x-1}$ možností pre zvyšné. $\leadsto x2^{x-1}$
        \item[Ani jedna charakteristika neobsahuje rameno s hodnotou $\bar{1}$:] Máme $2^x$ možností pre farby
        ramien. $\leadsto 2^x$.
    \end{description}

    Pokiaľ má vrchol $u$ charakteristiku $0$, máme dve základné možnosti. Ak má po prechode tiež charakteristiku $0$,
    máme presne rovnako veľa možností, ako keď mal vrchol $u$ starú charakteristiku $1$. Ak má po prechode
    charakteristiku $\bar{1}$, máme rovnako veľa možností, ako keď mal vrchol $u$ starú charakteristiku $\bar{1}$.

    Sčítaním a úpravou dostaneme:
    $$2(2^x + x(x-1)2^{x-2} + x2^{x-1} + x2^{x-1} + 2^x) = 2^{x+1}\left(2 + x + \frac{x(x-1)}{4}\right).$$
\end{proof}
\begin{dosl}
    \label{hviezda-odhad}
    Nech $H$ je netriviálna hviezda s $m$ vrcholmi a nech $p$ je jej celkový počet prechodov. Potom platí $\sqrt[m]{p} \leq
    \sqrt{12}$.
\end{dosl}
\begin{proof}
    Podľa predošlej lemy má hviezda s $m$ vrcholmi presne $p = 2^m(2 + m-1 + \frac{(m-1)(m-2)}{4})$. Pre $m \ge 2$
    platí $2 + m-1 + \frac{(m-1)(m-2)}{4} \leq m^2$. Zaujíma nás hodnota $\sqrt[m]{p}$, ktorej horný odhad je
    $\sqrt[m]{2^m m^2} = 2 m\strut^{\frac{2}{m}}$.

    Funkcia $m\strut^{\frac{2}{m}}$ je klesajúca pre $m \ge e$ a pre
    $m = 8$ je hodnota výrazu $2m\strut^{\frac{2}{m}}$ menšia ako $\sqrt{12}$. Pre $m=2$ je hodnota $p$ presne $12$, čiže
    $\sqrt[m]{p} = \sqrt{12}$. Pre $2 \leq m \leq 7$ uvedieme tabuľku približných hodnôt pôvodnej funkcie
    $\sqrt[m]{2 + m-1 + \frac{(m-1)(m-2)}{4}}$, čím zavŕšime dôkaz:

    \begin{tabular}{| l | c | c | c | c | c | c |} \hline
        $m =            $&$ 2 $&$ 3 $&$ 4 $&$ 5 $&$ 6 $&$ 7 $ \\ \hline
        $\sqrt[m]{p} =  $&$ 3.4641 $&$ 3.30193 $&$ 3.19344 $&$ 3.10369 $&$ 3.02617 $&$ 2.95854$ \\ \hline
    \end{tabular}
\end{proof}
\begin{dosl}
    Ľubovoľný graf $G$ s $n$ vrcholmi a bez izolovaných vrcholov má nanajvýš $12\strut^{\frac{n}{2}}$ prechodov.
\end{dosl}
\begin{proof}
    Nech $K_1 \ldots K_m$ sú komponenty grafu $G$. Pre každý z nich existuje pokrytie hviezdami, ktoré
    sa neprekrývajú. Keď zjednotíme pokrytie hviezdami pre každý komponent, dostaneme pokrytie hviezdami
    grafu $G$. Hviezdy v tomto pokrytí označíme $H_1 \ldots H_k$ a ich počty vrcholov označíme $n_1 \ldots n_k$
    v tomto poradí.

    Keď z grafu odstránime nejakú hranu, všetky pôvodné prechody ostanú platné. To znamená, že graf zložený
    z hviezd $H_1 \ldots H_k$ má aspoň toľko prechodov, ako graf $G$. Každá dvojica rôznych hviezd je nezávislá,
    preto počet prechodov na grafe zloženom z hviezd $H_1 \ldots H_k$ je presne $\prod \limits_{i=1}^{k} pr(H_i)$.
    Nakoniec s použitím predošlého dôsledku \ref{hviezda-odhad} odvodíme tvrdenie:

    $$ pr(G) \leq \prod \limits_{i=1}^{k} pr(H_i) \leq \prod \limits_{i=1}^{k} \sqrt{12}^{n_i} = \sqrt{12}\rule{0pt}{17pt}^{\left(\sum \limits_{i=1}^{k} n_i \right)} = \sqrt{12}^{n} = 12^{\frac{n}{2}}.$$
\end{proof}

\subsection{Časová zložitosť algoritmu}

Z definície množín $C$ a $D$ v algoritme musí platiť, že každý vrchol v $C$ má suseda v $D$ a taktiež
každý sused v $D$ má suseda v $C$. To znamená, že podgraf grafu $G$ indukovaný množinou $C \cup D$ nemá
izolované vrcholy a teda má nanajvýš $12\strut^{\frac{c+d}{2}}$ prechodov. Teraz máme pripravené podklady
pre horný odhad časovej zložitosti algoritmu.

Komponentov grafu $G - E$ je nanajvýš $n$ a každý z nich má nanajvýš $\frac{2n}{3}$ vrcholov. Pre každý
z nanajvýš $12\strut^{\frac{n}{8}}$ prechodov a pre každý z nanajvýš $n$ komponentov budeme počítať množinu
charakteristík, ktorá vznikne pridaním novej farby v čase $O^*\left(2.65\strut^{\frac{2n}{3}}\right)$. Pri každej
kombinácii prechodu a komponentu dostaneme množinu nanajvýš $O^*\left(2.65\strut^{\frac{2n}{3}}\right)$ charakteristík.
Pre každú z nich v polynomiálnom čase overíme, či je konzistentná s prechodom separátora.

Časová zložitosť tohto algoritmu je teda zhora ohraničená na $O^*\left(2.65\strut^{\frac{2n}{3}} \cdot 12\strut^{\frac{n}{8}}\right) = O^*(2.613^n)$.
