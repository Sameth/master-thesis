\chapter{TODO: Kapitola o zmensovani a rozoberani "specialnych" pripadov}

TODO: Zhrnutie

\section{TODO: Orezavanie mostov}

%Časová zložitosť algoritmu Junosza-Szaniawski je principiálne veľká na grafoch, v ktorých
%existuje veľa rôznych čiastočných $L(2,1)$ ofarbení. Ako sme videli v predošlej kapitole,
%spomedzi súvislých grafov majú najviac čiastočných ofarbení stromy. Na dostatočne jednoduchých
%triedach grafov však existujú polynomiálne algoritmy, ktoré riešia problém $L(2,1)$-farbenia.

Prvý typ ``jednoduchosti'' grafu, ktorú vieme využiť pri hľadaní rýchlejšieho algoritmu,
zodpovedá hranovej súvislosti grafu. Vezmime si nejaký hranový separátor $S_E$ grafu $G$. 
Všetkým vrcholom, ktoré sú incidentné s niektorou hranou separátora, zafixujeme nejaké farby.
Ďalej si vezmime komponenty súvislosti grafu $G - S_E$.

Ak v každom komponente súvislosti nájdeme (nezávisle) nejaké $L(2,1)$-farbenie
konzistentné s ohodnotením separátora, spojením farbení pre každý podgraf dostaneme
$L(2,1)$-farbenie celého grafu $G$. Toto tvrdenie si formálne zhrnieme v nasledujúcej leme.

\begin{lema}
    Nech $S_E \subseteq E(G)$ je hranový separátor grafu $G$,
    nech $S_V \subseteq V(G)$ je množina všetkých vrcholov, ktoré sú incidentné s niektorou hranou
    v $S_E$, nech $G_1, G_2, \cdots G_k$ sú komponenty súvislosti grafu $G - S_E$.

    Nech $f_S: S_V \to \mathbb{N}$ je čiastočné $L(2,1)$-farbenie množiny $S_V$ v grafe $G$,
    nech $f_1, f_2, \ldots, f_k$ sú čiastočné $L(2,1)$-farbenia množín $V(G_1) \cup S_V,\ 
    V(G_2) \cup S_V,\  \ldots, V(G_k) \cup S_V$ v grafe $G$ a nech platí

    $$ \forall i \in {1 \ldots k},\  \forall v \in S_V: f_i(v) = f_S(v) .$$

    Potom zobrazenie $\omega: V(G) \to \mathbb{N}$ definované nasledovne:

    \[ \omega(v) =
    \begin{cases}
        f_S(v), & v \in S_V \\
        f_i(v), & v \in V(G_i) - S_V
    \end{cases}
    \]

    tvorí $L(2,1)$-farbenie grafu $G$.

\end{lema}

\begin{proof}
    Zobrazenie $\omega$ je dobre definované, lebo každý vrchol patrí do práve jedného komponentu grafu
    $G - S_E$, zároveň každému vrcholu v $G$ priradí hodnotu. Potrebujeme teda overiť podmienky $L(2,1)$-farbenia
    pre každú dvojicu vrcholov $u, v \in V(G)$. Ďalej rozoberieme niekoľko prípadov podľa príslušnosti $u, v$:

    \begin{description}
        \item[Oba vrcholy patria do množiny $S_V$:] Z definície $\omega$ platí $\omega(u) = f_S(u), \omega(v) = f_S(v)$.
        Keďže $f_S$ je čiastočné $L(2,1)$-farbenie množiny $S_V$, vrcholy $u, v$ musia spĺňať podmienky $L(2,1)$-farbenia.

        \item[Vrchol $u$ patrí do $S_V$, vrchol $v$ do $V(G_i) - S_V$ pre niektoré $i$:] Z definície $\omega$ platí
        $\omega(u) = f_S(u) = f_i(u)$ a $\omega(v) = f_i(v)$. Keďže $f_i$ je čiastočné $L(2, 1)$-farbenie, vrcholy
        $u, v$ musia spĺňať podmienky $L(2,1)$-farbenia.
        
        \item[Oba vrcholy patria do toho istého komponentu $G_i$:] Z definície
        $\omega$ platí $\omega(u) = f_i(u), \omega(v) = f_i(v)$. Keďže $f_i$ je čiastočné $L(2,1)$-farbenie, vrcholy
        $u, v$ musia spĺňať podmienky $L(2,1)$-farbenia.

        \item[Vrchol $u$ patrí do komponentu $G_i$, vrchol $v$ do komponentu $G_j$, $i \neq j$:] Keďže vrcholy patria
        do rôznych komponentov grafu $G - S_E$, každá cesta medzi $u, v$ obsahuje aspoň jednu hranu z $S_E$. Špeciálne
        to platí pre každú najkratšiu cestu medzi $u$ a $v$. Keďže $u, v \notin S_V$, každá najkratšia cesta medzi $u$
        a $v$ musí pozostávať z neprázdnej cesty z $u$ do niektorého vrchola v $S_V$, hrany v $S_E$ a z neprázdnej
        cesty z vrchola v $S_V$ do vrchola $v$. To znamená, že vrcholy $u$ a $v$ sú vzdialené aspoň $3$, teda spĺňajú
        podmienky $L(2,1)$-farbenia.
    \end{description}

    Každá dvojica vrcholov spĺňa obmedzenia $L(2,1)$-farbenia, a preto $\omega$ je $L(2,1)$-farbením grafu $G$. \qedhere

\end{proof}

\begin{dosl}
    Nech $e \in E(G)$ je most grafu $G$, nech $u, v$ sú vrcholy hrany $e$, nech graf $G - \{e\}$ pozostáva z dvoch
    komponentov $G_1$ a $G_2$ veľkosti aspoň $2$. Potom je graf $G$ $k$-$L(2,1)$-zafarbiteľný práve vtedy, ak
    existujú hodnoty $a, b$ také, že $0 \leq a, b \leq k, \left| a - b \right| \ge 2$ a existujú $k$-$L(2,1)$-farbenia $f_1, f_2$ grafov
    $G_1 \cup \{v, u, e\}$ a $G_2 \cup \{v, u, e\}$, pre ktoré platí $f_1(u) = f_2(u) = a \wedge f_1(v) = f_2(v) = b$.
\end{dosl}

\begin{proof}
    $\boxed{\Rightarrow}$
        Ak je graf $G$ $k$-$L(2,1)$-zafarbiteľný, existuje jeho $k$-$L(2,1)$-farbenie $f$. Zobrazenia $f_1, f_2$ môžeme
        definovať ako zúženie zobrazenia $f$ na príslušné množiny vrcholov. Hodnoty $a, b$ určíme ako $f(u)$, resp. $f(v)$.

    $\boxed{\Leftarrow}$
        Tvrdenie je inštanciou predošlej lemy: $S_E = \{e\},\ S_V = \{u,v\},\ f_S(u) = a,\ f_S(v) = b$. Keďže $f_1$
        je $k$-$L(2,1)$-farbenie grafu $G_1 \cup \{v, u, e\}$, ide o čiastočné $L(2,1)$-farbenie v jeho nadgrafe $G$.
        Inak povedané, ide o číastočné $L(2,1)$-farbenie množiny $V(G_1) \cup \{u, v\}$. Analogicky je $f_2$ čiastočné
        farbenie množiny $V(G_2) \cup \{u,v\}$. Obe farbenia používajú hodnoty medzi $0$ a $k$ vrátane, čiže ich
        spojením vznikne tiež $k$-$L(2,1)$-farbenie.
\end{proof}

Tento dôsledok nám vlastne dáva spôsob, ktorým vieme rozdeliť rozhodovanie
problému $k$-$L(2,1)$-farbenia grafu $G$ na dve nezávislé časti v prípade,
že $G$ obsahuje netriviálny most -- most, ktorého odstránením dostaneme
dva komponenty súvislosti s aspoň dvomi vrcholmi.

Pre každé možné ohodnotenie vrcholov mosta vyskúšame nezávisle ofarbiť oba
komponenty. K obom komponentom potrebujeme pripojiť aj most a jeho druhý koniec. Za cenu
skúšania zhruba $k^2$ možností ofarbenia mostových vrcholov dostaneme dva nezávislé
problémy.

Intuitívne platí, že rozdelenie na dva podproblémy
najviac zlepší časovú zložitosť vtedy, keď majú oba komponenty zhruba rovnakú veľkosť.
Ak bude jeden komponent príliš malý, môžeme si takýmto rozdelením časovú zložitosť zhoršiť:
Za zanedbateľné zmenšenie problému zaplatíme tým, že ho budeme musieť riešiť viackrát.

V ďalšej časti rozoberieme, ako sa vieme vysporiadať s problémom malých komponentov a
aký výsledok z toho dostaneme.

\subsection{Mostová veta}


