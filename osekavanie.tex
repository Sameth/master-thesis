\chapter{TODO: Kapitola o zmensovani a rozoberani "specialnych" pripadov}

TODO: Zhrnutie

\section{TODO: Orezavanie mostov}

%Časová zložitosť algoritmu Junosza-Szaniawski je principiálne veľká na grafoch, v ktorých
%existuje veľa rôznych čiastočných $L(2,1)$ ofarbení. Ako sme videli v predošlej kapitole,
%spomedzi súvislých grafov majú najviac čiastočných ofarbení stromy. Na dostatočne jednoduchých
%triedach grafov však existujú polynomiálne algoritmy, ktoré riešia problém $L(2,1)$-farbenia.

Prvý typ ``jednoduchosti'' grafu, ktorú vieme využiť pri hľadaní rýchlejšieho algoritmu,
zodpovedá hranovej súvislosti grafu. Vezmime si nejaký hranový separátor $S_E$ grafu $G$. 
Všetkým vrcholom, ktoré sú incidentné s niektorou hranou separátora, zafixujeme nejaké farby.
Ďalej si vezmime komponenty súvislosti grafu $G - S_E$. Každá dvojica vrcholov, ktoré sú
v rôznych komponentoch a ešte nemajú priradenú farbu, má vzdialenosť aspoň $3$.

Ak v každom komponente súvislosti nájdeme (nezávisle) nejaké $L(2,1)$-farbenie
konzistentné s ohodnotením separátora, dostaneme tým nejaké správne $L(2,1)$ farbenie
celého grafu $G$. Toto tvrdenie si formálne zhrnieme v nasledujúcej leme.

\begin{lema}
    FIXME: dopisat lemu %TODO!!!!
    Nech $S_E \subseteq E(G)$ je hranový separátor grafu $G$,
    nech $S_V \subseteq V(G)$ je množina všetkých vrcholov, ktoré sú incidentné s niektorou hranou
    v $S_E$, nech $G_1, G_2, \cdots G_k$ sú komponenty súvislosti grafu $G - S_E$. 
\end{lema}

\begin{defn}

\end{defn}

\subsection{TODO}

