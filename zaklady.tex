\chapter{Základné pojmy a predošlé výsledky}

V tejto kapitole si zavedieme pojmy a označenia, ktoré budeme používať
vo zvyšku práce. Taktiež si podrobne popíšeme predošlé výsledky o
$L(2,1)$-farbeniach grafov, ktoré budú dôležité pre túto prácu.

Pokiaľ nebude povedané inak, grafy spomínané v tejto práci budú neorientované,
súvislé a jednoduché, bez slučiek a násobných hrán. Množinu vrcholov budeme
označovať $V(G)$ a množinu hrán $E(G)$. Vzdialenosť dvoch vrcholov $u$ a $v$
v grafe budeme označovať $d(u,v)$. 

Ako prvé si poriadne
zadefinujeme $L(2,1)$-farbenie grafu a problémy s ním súvisiace.

\begin{defn}
$L(2,1)$-farbenie grafu $G$ je zobrazenie $\omega$, ktoré každému vrcholu priradí
nezáporné celé číslo spĺňajúc nasledujúce podmienky:

\begin{itemize}
\item $d(u, v) = 1 \Rightarrow \left| \omega(u) - \omega(v) \right| \ge 2$
\item $d(u, v) = 2 \Rightarrow \omega(u) \neq \omega(v)$
\end{itemize}
\end{defn}

Pri obyčajných vrcholových, či hranových farbeniach grafov skúmame, aký
najmenší počet farieb potrebujeme pre platné ofarbenie daného grafu. Podobne
si vieme pri $L(2,1)$-farbeniach všímať, aké najväčšie číslo používajú. Toto
číslo budeme nazývať \emph{rozpätím $L(2,1)$-farbenia}. Najmenšie možné rozpätie
$L(2,1)$-farbenia grafu $G$ budeme nazývať \emph{$L(2,1)$-farbiace číslo} a
označovať $\lambda(G)$.

Pod problémom $k-L(2,1)$-farbenia budeme rozumieť problém existencie farbenia
s rozpätím nanajvýš $k$. Podobne vieme zadefinovať problém $L(2,1)$-farbenia ako
hľadanie $\lambda(G)$. Keďže pre každý graf existuje triviálne ofarbenie
používajúce párne čísla od $0$ po $2n - 2$, z hľadiska výpočtovej zložitosti
sú tieto problémy takmer totožné.

Keďže ide o modifikáciu problému farbenia grafov, je očakávateľné, že tento
problém bude $\mathcal{NP}$-ťažký. Tento výsledok dokázali už Griggs a Yeh
\cite{griggs_yeh_tree}. V tom istom článku taktiež ukázali ekvivalenciu
nami definovaného problému so zovšeobecnením problému na reálne čísla.


\section{Farbenie špeciálnych grafov}



\section{Farbenie všeobecných grafov}
