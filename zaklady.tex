\chapter{Základné pojmy a predošlé výsledky}

V tejto kapitole si zavedieme pojmy a označenia, ktoré budeme používať
vo zvyšku práce. Taktiež si podrobne popíšeme predošlé výsledky o
$L(2,1)$-farbeniach grafov, ktoré budú dôležité pre túto prácu.

Pokiaľ nebude povedané inak, grafy spomínané v tejto práci budú neorientované,
súvislé a jednoduché, bez slučiek a násobných hrán. Množinu vrcholov budeme
označovať $V(G)$ a množinu hrán $E(G)$. Vzdialenosť dvoch vrcholov $u$ a $v$
v grafe budeme označovať $d(u,v)$. 

Ako prvé si poriadne
zadefinujeme $L(2,1)$-farbenie grafu a problémy s ním súvisiace.

\begin{defn}
$L(2,1)$-farbenie grafu $G$ je zobrazenie $\omega$, ktoré každému vrcholu priradí
nezáporné celé číslo spĺňajúc nasledujúce podmienky:

\begin{itemize}
\item $d(u, v) = 1 \Rightarrow \left| \omega(u) - \omega(v) \right| \ge 2$
\item $d(u, v) = 2 \Rightarrow \omega(u) \neq \omega(v)$
\end{itemize}
\end{defn}

Pri obyčajných vrcholových, či hranových farbeniach grafov skúmame, aký
najmenší počet farieb potrebujeme pre platné ofarbenie daného grafu. Podobne
si vieme pri $L(2,1)$-farbeniach všímať, aké najväčšie číslo používajú. Toto
číslo budeme nazývať \emph{rozpätím $L(2,1)$-farbenia}. Najmenšie možné rozpätie
$L(2,1)$-farbenia grafu $G$ budeme nazývať \emph{$L(2,1)$-farbiace číslo} a
označovať $\lambda(G)$. Dôležité je všimnúť si, že $L(2,1)$-farbenie s
rozpätím $k$ môže používať až $k+1$ rôznych hodnôt.

Pod problémom $k$-$L(2,1)$-farbenia budeme rozumieť problém existencie farbenia
s rozpätím nanajvýš $k$. Podobne vieme zadefinovať problém $L(2,1)$-farbenia ako
hľadanie $\lambda(G)$. Keďže pre každý graf existuje triviálne ofarbenie
používajúce párne čísla od $0$ po $2n - 2$, z hľadiska výpočtovej zložitosti
sú tieto problémy takmer totožné.

Jedným z dôvodov, prečo je $L(2,1)$-farbeniam venovaná pozornosť, je ich uplatnenie
pri prideľovaní frekvenčných pásiem vysielacím staniciam. Kvôli interferencii
elektromagnetických vysielačov a prijímačov na podobných frekvenciách totiž
musia veľmi blízke vysielače dostať veľmi rozdielne pásmo, ale viac vzdialeným
vysielačom stačí menší rozdiel.

Aj keď je náš problém formulovaný jazykom celých čísel, už v prvých článkoch
venujúcich sa tejto problematike bola dokázaná ekvivalencia
nami definovaného problému so zovšeobecnením problému na reálne čísla\cite{griggs_yeh_tree}.

Keďže ide o modifikáciu problému farbenia grafov, je očakávateľné, že tento
problém bude $\mathcal{NP}$-ťažký, čo sa podarilo dokázať v tom istom článku.
Taktiež sa postupom času dokázala zložitosť problému aj na niektorých obmedzených
triedach grafov, napr. planárnych, bipartitných, dokonca aj na grafoch s
priemerom $2$\cite{color_survey}.

Ďalšou oblasťou skúmania boli dolné a horné ohraničenia na hodnotu $\lambda(G)$.
Vhodnou charakteristikou sa ukázal byť maximálny stupeň grafu. Pre všeobecné
grafy bolo najprv dokázané horné ohraničenie $\lambda(G) \leq \Delta^2 + 2\Delta$
pomocou jednoduchého pažravého priradenia\cite{griggs_yeh_tree},
ktoré bolo neskôr vylepšené na $\lambda(G) \leq \Delta^2 + \Delta$ s použitím
šikovnejšieho priradzovania\cite{chang_kuo}.

Pre špeciálne triedy grafov sa podarilo dokázať silnejšie ohraničenia. Napríklad
pre grafy s priemerom $2$ platí a je tesný odhad $\lambda(G) \leq \Delta^2$\cite{griggs_yeh_tree}
. Ďalej pre chordálne grafy platí odhad $\lambda(G) \leq \frac{\left( \Delta + 3 \right)^2}{4}$\cite{griggs_yeh_tree}
.

Medzi ďalšie triedy grafov, na ktorých sú dokázané tesnejšie obmedzenia na $L(2,1)$-farbiace
číslo, patria aj stromy s obmedzením $\Delta + 1 \leq \lambda(G) \leq \Delta + 2$\cite{griggs_yeh_tree},
kaktusy s obmedzením $\Delta + 1 \leq \lambda(G) \leq \Delta + 3$ a vonkajšie planárne
grafy s obmedzením $\Delta + 1 \leq \lambda(G) \leq \Delta + 8$\cite{outer_planar_bound}.

Griggs a Yeh taktiež vyslovili hypotézu, že v každom grafe platí $\lambda(G) \leq \Delta^2$,
ktorá dodnes nebola vyriešená.

Ďalej sa pozrieme na zopár tried grafov, na ktorých vieme riešiť problém $L(2,1)$-farbenia
v polynomiálnom čase a popíšeme si zodpovedajúce algoritmy.

\section{Polynomiálne farbenie špeciálnych grafov}

Prvou triedou grafov, pre ktoré bol objavený polynomiálny algoritmus na riešenie
problému $k$-$L(2,1)$-farbenia, sú stromy\cite{chang_kuo}. Tento algoritmus si
podrobne popíšeme.

\subsection{Chang-Kuo algoritmus}

Vstupom pre algoritmus je strom $T$ zakorenený v listovom vrchole so synom $r$ a číslo $k$.
Tento algoritmus zistí, či existuje $L(2,1)$-farbenie grafu $T$ s rozpätím $k$.

Pre ľubovoľný
vrchol $v$ zadefinujeme $T(v)$ ako podstrom stromu $T$ zakorenený vo vrchole $v$. Taktiež
zadefinujeme $T'(v')$ ako $T(v)$ s novým vrcholom $v'$, ktorý je spojený iba s $v$, čiže
$$ G\left(V(T(v)) \cup \{ v' \}, E(T(v)) \cup \{ (v', v)\} \right) $$
kde $v'$ je nový vrchol nevyskytujúci sa v $T(v)$.

Základnou myšlienkou algoritmu je konštruovať množiny $S(T(v))$ definované nasledovne:
$$S(T(v)) = \{ (a, b) | \textrm{ existuje } L(2,1) \textrm{ farbenie } f \textrm{ grafu } T'(v') \textrm{ také, že } f(v) = a \textrm{ a } f(v') = b\}$$

Pre ľubovoľný listový vrchol $v_l$ ľahko vidíme, že 
$$S(T(v_l)) = \{ (a, b) | a \leq k \wedge b \leq k \wedge |a - b| \ge 2\}$$

Pre vnútorný vrchol $v$ so synmi $v_1, v_2, \ldots v_q$ vieme pre ľubovoľné $a, b, |a - b| \ge 2$ overiť,
či $(a, b) \in S(T(v))$ nasledovne:

\begin{enumerate}
\item Zostrojíme bipartitný graf $G_{a,b}(v)$, ktorého jednou partíciou budú vrcholy $v_1, v_2 \ldots v_q$ a
druhou partíciou hodnoty množiny $\{0, 1, \ldots k\}$ a ktorého množinu hrán zostrojíme nasledovne:
$$E(G_{a,b}) = \{ (v_i, x) | x \neq b \wedge (a, x) \in S(T(v_i))\}$$

\item Nájdeme najväčšie párenie na grafe $G_{a,b}$.

\item Ak nájdené párenie má veľkosť $q$, potom $(a,b) \in S(T(v))$, inak $(a,b) \notin S(T(v))$.
\end{enumerate}

Do pozornosti dávame skutočnosť, že nájdené párenie zodpovedá priradeniu hodnôt vrcholom
$v_1, \ldots v_q$, pri ktorom je platné ohodnotenie vrcholov $v$ a $v'$ hodnotami $a$ a $b$.

Nakoniec $L(2,1)$-farbenie stromu $T$ s rozpätím $k$ existuje práve vtedy, keď je množina
$S(T(r))$ neprázdna.

\subsection{Kaktusy a vonkajšie planárne grafy}



\section{Farbenie všeobecných grafov}
