\chapter{Základné pojmy a predošlé výsledky}

V tejto kapitole si zavedieme pojmy a označenia, ktoré budeme používať
vo zvyšku práce. Taktiež si podrobne popíšeme predošlé výsledky o
$L(2,1)$-farbeniach grafov, ktoré budú dôležité pre túto prácu.

Pokiaľ nebude povedané inak, grafy spomínané v tejto práci budú neorientované,
súvislé a jednoduché, bez slučiek a násobných hrán. Množinu vrcholov budeme
označovať $V(G)$ a množinu hrán $E(G)$. Vzdialenosť dvoch
vrcholov $u$ a $v$ v grafe budeme označovať $d(u,v)$.

Ako prvé si poriadne
zadefinujeme $L(2,1)$-farbenie grafu a problémy s ním súvisiace.

\begin{defn}
$L(2,1)$-farbenie grafu $G$ je zobrazenie $\omega$, ktoré každému vrcholu priradí
nezáporné celé číslo spĺňajúc nasledujúce podmienky:

\begin{itemize}
\item $d(u, v) = 1 \Rightarrow \left| \omega(u) - \omega(v) \right| \ge 2$
\item $d(u, v) = 2 \Rightarrow \omega(u) \neq \omega(v)$
\end{itemize}
\end{defn}




\section{Farbenie špeciálnych grafov}



\section{Farbenie všeobecných grafov}
