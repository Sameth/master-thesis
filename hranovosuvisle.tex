\chapter{Experiment na $2$-hranovo súvislých grafoch}

V prvej kapitole sme sa venovali konštrukcii algoritmu, ktorý dokáže z efektívneho riešenia
problému $L(2,1)$-farbenia na chlpatých $2$-hranovo-súvislých grafoch vyskladať efektívne
riešenie problému $L(2,1)$-farbenia na všeobecných grafoch.

Najefektívnejšie riešenie problému $L(2,1)$-farbenia má časovú zložitosť zhora ohraničenú
počtom vlastných párov na súvislých grafoch. V tejto kapitole sa budeme venovať
experimentálnemu odhadu počtu vlastných párov na $2$-hranovo súvislých grafoch.

Cieľom experimentu bolo odhaliť, aký typ $2$-hranovo súvislých grafov má najviac
vlastných párov.

\section{Konštrukcia $2$-hranovo súvislých grafov}

Najprv popíšeme a dokážeme spôsob, ktorým vieme skonštruovať všetky $2$-hranovo
súvislé grafy. Bude sa nápadne ponášať na konštrukciu $2$-súvislých grafov. Každý
$2$-hranovo súvislý graf sa dá vytvoriť z kružnice pridávaním $H$-ciest do existujúceho
grafu \cite{diestel_graph}. Tento postup vieme rozšíriť na $2$-hranovo súvislé grafy, ak
dovolíme okrem $H$-ciest pridávať aj $H$-kružnice.

\begin{defn}
    Nech $H$ je graf. Cestu $P$ budeme nazývať $H$-cesta, ak prienik $P$ a $H$ tvoria
    práve koncové vrcholy $P$.

    Kružnicu $C$ budeme nazývať $H$-kružnica, ak prienik $C$ a $H$ tvorí
    práve jeden vrchol v $C$.
\end{defn}

\begin{veta}
    Každý $2$-hranovo súvislý graf $G$ sa dá vytvoriť z kružnice postupným pridávaním
    $H$-ciest a $H$-kružníc do grafov $H$, ktoré sme už skonštruovali.
\end{veta}
\begin{proof}
    Najprv dokážeme, že $G$ obsahuje kružnicu ako podgraf. Graf $G$ je $2$-hranovo súvislý,
    preto pre ľubovoľnú hranu $(uv) = e \in E(G)$ platí, že graf $G - e$ je súvislý. Nájdeme
    cestu $P$ medzi vrcholmi $u$ a $v$ v grafe $G - E$. Cesta $P$ doplnená o hranu $e$ tvorí
    kružnicu.

    Ďalej budeme postupovať sporom, graf $G$ sa nedá zostrojiť popísaným spôsobom. Graf $G$
    obsahuje kružnicu a teda existuje nejaký podgraf $G$, ktorý sa dá týmto spôsobom
    skonštruovať. Označme $H$ ako najväčší (vzhľadom na inklúziu) takýto podgraf.

    Ak by existovala hrana $(uv) \in E(G) - E(H)$, ktorej koncové vrcholy $u, v$ ležia v
    $H$, tak $(uv)$ tvorí $H$-cestu a teda je v spore s maximálnosťou $H$. To znamená, že
    $H$ je indukovaný podgraf.
    
    Graf $G$ je súvislý a teda v ňom musí existovať hrana
    medzi nejakou dvojicou vrcholov $u, v$, kde $u \in V(G) - V(H)$ a $v \in V(H)$.
    Z predpokladu, že $G$ je $2$-hranovo súvislý vieme, že $G - (uv)$ je súvislý a teda
    v ňom existuje nejaká cesta $P$ medzi vrcholom $u$ a $v$. Nech $w$ je posledný vrchol
    na ceste $P$ z množiny $V(G) - V(H)$ a $P_w$ je časť cesty $P$ od vrchola $w$ do $v$.
    Cesta $P_w$ s hranou $(uv)$ tvorí $H$-cestu, ak $w \neq u$, alebo $H$-kružnicu, ak $w = u$.

    To znamená, že pridaním cesty $P_w$ s hranou $(uv)$ do grafu $H$ dostaneme väčší graf,
    ktorý vieme skonštruovať popísaným postupom. To je však v spore s predpokladom, že $H$ je
    maximálny podgraf, ktorý vieme týmto postupom skonštruovať.
\end{proof}

Za zmienku stojí fakt, že dôkaz vety nám dáva spôsob, ako pre ľubovoľný graf určiť
postupnosť operácií, ktorými ho vieme vytvoriť.

Množinu všetkých $2$-hranovo súvislých grafov s daným počtom vrcholov teda vieme skonštruovať
nasledovne: Najprv určíme, aký veľký je prvý cyklus a pre každú túto možnosť spustíme
rekurzívnu procedúru, ktorá pridáva $H$-kružnice a $H$-cesty.

Rekurzívna procedúra dostane ako parameter nejaký $2$-hranovo súvislý graf $H$ a počet
vrcholov $k$, ktoré treba ešte do grafu pridať.

Pridávanie $H$-ciest bude fungovať nasledovne: Pre každý možný výber dvojice rôznych
vrcholov $u, v$ a každú možnosť počtu nových vrcholov $p \leq k$ vytvorí graf $H'$,
Graf $H'$ vznikne z grafu $H$ pridaním cesty s $p$ novými vrcholmi medzi vrcholy $u, v$.
Nakoniec rekurzívne spustí túto procedúru s parametrami $H'$ a $k - p$.

Pridávanie $H$-cyklov bude fungovať podobne: Pre každý vrchol $v$ a pre každý možný
počet nových vrcholov $2 \leq p \leq k$ skonštruujeme graf $H'$. Graf $H'$ vznikne z
grafu $H'$ pridaním cyklu s $p+1$ vrcholmi, kde jeden z nich je $v$. Na graf $H'$
rekurzívne spustíme procedúru s parametrami $H'$ a $k - p$.

Takto popísaný algoritmus pre konštrukciu $2$-hranovo súvislých grafov je nepraktický,
lebo väčšinu $2$-hranovo súvislých grafov vieme podľa predošlého postupu skonštruovať
viackrát. Pri skúmaní počtu vlastných párov nás nezaujíma označenie vrcholov. Preto stačí,
ak každý neoznačený $2$-hranovo súvislý graf skonštruujeme aspoň raz.

V ďalšej časti si popíšeme viacero spôsobov, ktorými vieme zredukovať množstvo vygenerovaných
grafov.

\section{Redukcia počtu grafov}


