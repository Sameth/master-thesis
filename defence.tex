\documentclass{beamer}
\usepackage[slovak]{babel}
\usepackage[utf8x]{inputenc}

\title[$L(2,1)$-farbenia grafov]{Farbenia grafov s obmedzeniami do vzdialenosti dva}
\author{Bc. Jaroslav Petrucha \\ školiteľ: RNDr. Michal Foríšek, PhD.}
\date{14. 6. 2018}

\begin{document}

\begin{frame}
\titlepage
\end{frame}

\begin{frame}{$L(2,1)$-farbenie}
    \begin{itemize}
        \item Ohodnotenie vrcholov prirodzenými číslami
        \begin{itemize}
            \item Susedné vrcholy majú veľký rozdiel
            \item Vzdialené vrcholy sa líšia
        \end{itemize}
    \end{itemize}
        FIXME: Obrázok!
\end{frame}

\begin{frame}{Súvisiace problémy}
    \begin{itemize}
        \item Skúmame rozsah priradených hodnôt
        \item Rozhodovacie aj optimalizačné problémy
        \begin{itemize}
            \item Stačí rozsah $k$?
            \item Aký je minimálny rozsah?
        \end{itemize}
        \item{$\mathcal{NP}$-ťažký na mnohých triedach}
        \begin{itemize}
            \item Planárne grafy
            \item FIXME: Treewidth 2
        \end{itemize}
    \end{itemize}
\end{frame}

\begin{frame}{Polynomiálny algoritmus}
    \begin{itemize}
        \item Špeciálne triedy grafov
        \begin{itemize}
            \item Stromy, cyklové stromy
            \item Vonkajšoplanárne grafy
        \end{itemize}
        \item Lokálne riešenie malých podproblémov
    \end{itemize}
\end{frame}

\begin{frame}{Všeobecný algoritmus}
    \begin{itemize}
        \item Dynamické programovanie
        \begin{itemize}
            \item FIXME: Časová zložitosť
            \item FIXME: Havet et al.
        \end{itemize}
        \item Množiny všetkých čiastočných farbení
        \begin{itemize}
            \item $O^*(2.6488^n)$
            \item FIXME: Junosza-Szaniawski et al.
        \end{itemize}
    \end{itemize}
\end{frame}

\begin{frame}{Vlastná práca - mosty}
    \begin{itemize}
        \item
    \end{itemize}
\end{frame}

\begin{frame}{Junosza-Szaniawski}
    \begin{itemize}
        \item 
    \end{itemize}
\end{frame}

\begin{frame}{Civitas v škatuľke}
%    \includegraphics[scale=0.7]{civitas-arch.pdf}
\end{frame}

\begin{frame}{Civitas v škatuľke - inicializácia}
    \begin{enumerate}
        \item Vedúci určí registračných aj výsledných sčítavačov podľa verejných kľúčov.
        \item Matrikár určí voličov, pre každého budú známe dva verejné kľúče: registračný a
        overovací.
        \item Výslední sčítavači spolu vygenerujú tajný súkromný a známy verejný kľúč $K_{V}$ k šifrovacej
        schéme.
        \item Registrační sčítavači pre každého voliča distribuovane vygenerujú jeho volebné oprávnenie (čerstvú
        správu) s verejnou a súkromnou podobou.
    \end{enumerate}
\end{frame}

\begin{frame}{Civitas v škatuľke - volenie}
    \begin{enumerate}
        \item Volič s registračným kľúčom získa od každého registračného sčítavača časť svojho
        oprávnenia.
        \item S pomocou overovacieho kľúča zistí, či je každý podiel vierohodný.
        \item Z podielov rekonštruuje oprávnenie.
        \item Odosiela svoj zašifrovaný hlas s oprávnením do ``volebnej urny''. 
        \item S hlasom posiela dôkaz, že je korektný a že pozná otvorený text hlasu a oprávnenia.
    \end{enumerate}
\end{frame}

\begin{frame}{Civitas v škatuľke - sčítanie}
    \begin{enumerate}
        \item Výslední sčítavači pozbierajú všetky hlasy.
        \item Skontrolujú sa dôkazy k hlasom, neplatné sa vyhodia.
        \item Prípadné duplicitné hlasy sa spracujú podľa vopred daných podmienok.
        \item Každý výsledný sčítavač aplikuje svoju vlastnú (náhodnú) permutáciu s prešifrovaním na
        hlasy a oprávnenia.
        \item Kvadraticky sa skontroluje oprávnenosť každého hlasu.
        \item Hlasy sa odšifrujú a zverejnia (oprávnenia nie).
    \end{enumerate}
\end{frame}

\begin{frame}{ElGamal - pripomenutie}
    \begin{itemize}
        \item Grupa $G$ rádu $q$ a generátor $g$
        \item Súkromný kľúč $x$, verejný kľúč $h = g^x$.
        \item Šifrovanie: náhodné $y$ a správa $m$, šifrovaná správa je $(c_1, c_2) = (g^y, m \cdot h^y)$.
        \item Dešifrovanie: $m = c_2 \cdot c_1^{-x}$.
        \item Prešifrovanie novým kľúčom: náhodné $r$, nová správa je $(g^r \cdot c_1, h^r \cdot c_2)$.

        \bigskip

        \item Homomorfná vlastnosť: $E(m_1 \cdot m_2) = E(m_1) \cdot E(m_2)$.
    \end{itemize}
\end{frame}

\begin{frame}{Dôkaz znalosti diskrétneho logaritmu}
    \begin{itemize}
        \item Alica dokazuje Bobovi, že pozná $x : h = g^x$.
    \end{itemize}
    \bigskip
    \begin{enumerate}
        \item Alica si zvolí náhodné $r$ a pošle Bobovi $t = g^r$.
        \item Bob si zvolí náhodné $c$ a pošle ho Alici.
        \item Alica pošle Bobovi $r = v - c \cdot x$.
        \item Bob overí $t = g^r \cdot y^c$.
    \end{enumerate}
    Veľmi podobne vieme dokázať, že dve dvojice prvkov $(g_i, g_i^{x_i})$ majú
    rovnaké $x$.
\end{frame}

\begin{frame}{Distribuovaný ElGamal}
    \begin{enumerate}
        \item Účastník si vygeneruje $x_i$ a $h_i = g^{x_i}$.
        \item Zverejní commit: $H(y_i)$.
        \item Po zverejnení všetkých ostatných commitov zverejní $y_i$ a dokáže znalosť logaritmu.
        \item Overí ostatné dôkazy a commity.
        \item Spoločný verejný kľúč je súčin jednotlivých verejných kľúčov, súkromný kľúč je súčet
        súkroných kľúčov.
    \end{enumerate}

    Pre dešifrovanie správy $(a,b)$ stačí zverejniť $a^{x_i}$ a dokázať ekvivalenciu diskrétnych
    logaritmov.
\end{frame}

\begin{frame}{Bezpečnosť}
    Schéma Civitas je overiteľná a zabraňuje ovplyvňovaniu voliča, za predpokladov:
    \begin{enumerate}
        \item Útočník nevie plne odsimulovať voliča,
        \item Každý volič verí aspoň jednému registračnému sčítavačovi a má s ním dôverné spojenie.
        \item Volič verí zariadeniu, s ktorým hlasuje.
        \item Všetka komunikácia voliča je anonymná.
        \item Aspoň jedna hlasovacia urna, ktorú volič použil, sa správa korektne.
        \item Aspoň jeden výsledný sčítavač nie je útočníkom.
        \item Rozhodovací Diffie-Hellman problém aj RSA-problém sú výpočtovo náročné, SHA-256
        je náhodné orákulum.
    \end{enumerate}
\end{frame}
\end{document}
