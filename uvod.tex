\chapter*{Úvod} % chapter* je necislovana kapitola
\addcontentsline{toc}{chapter}{Úvod} % rucne pridanie do obsahu
\markboth{Úvod}{Úvod} % vyriesenie hlaviciek

Farbenie grafov je známy a pomerne dobre preštudovaný ťažký problém na grafoch.
Jedným zo zovšeobecnení farbenia grafu je $L(2,1)$-farbenie grafu, v ktorom
vrcholom priradzujeme prirodzené čísla. Požadujeme pri tom, aby susedné
vrcholy mali priradené čísla s rozdielom aspoň $2$ a vrcholy so spoločným
susedom mali priradené rôzne čísla.

Jedným z dôvodov, prečo je $L(2,1)$-farbeniam venovaná pozornosť, je ich uplatnenie
pri prideľovaní frekvenčných pásiem vysielacím staniciam. Kvôli interferencii
elektromagnetických vysielačov a prijímačov na podobných frekvenciách totiž
musia veľmi blízke vysielače dostať veľmi rozdielne pásmo, ale viac vzdialeným
vysielačom stačí menší rozdiel.

Dôležitou charakteristikou $L(2,1)$-farbenia
je jeho rozpätie -- najväčšie číslo, ktoré je priradené niektorému vrcholu. Väčšina
problémov ohľadom $L(2,1)$-farbení sa týka práve tohto rozpätia. Jedným zo základných
problémov je pre daný graf a danú hodnotu $k$ povedať, či existuje $L(2,1)$-farbenie
tohto grafu s rozpätím $k$.

Problémy $L(2,1)$-farbenia formulujeme v jazyku prirodzených čísel. Ukazuje sa,
že toto zjednodušenie z reálneho sveta je postačujúce. Už v prvých prácach venujúcich sa tejto problematike
bola dokázaná ekvivalencia problémov definovaných na prirodzených číslach
so zovšeobecnením na reálne čísla \cite{griggs_yeh_tree}.

Keďže ide o problém príbuzný farbeniam grafov, je očakávateľné, že
bude NP-ťažký, čo sa podarilo dokázať v tom istom článku.
Taktiež sa postupom času dokázala zložitosť problému aj na niektorých obmedzených
triedach grafov, napr. planárnych, bipartitných, na grafoch s
priemerom $2$ \cite{color_survey}.

Pre niektoré triedy grafov boli objavené algoritmy, ktoré hľadajú minimálne rozpätie
$L(2,1)$-farbenia v polynomiálnom čase. Prvým výsledkom bol polynomiálny algoritmus
na stromoch \cite{chang_kuo}. Ďalšiu triedu grafov s polynomiálnym algoritmom tvoria
cyklové stromy \cite{kaktusy} -- grafy, v ktorých sú všetky kružnice vrcholovo disjunktné.

Druhou oblasťou skúmania sú algoritmy, ktoré riešia problém na všeobecných grafoch. Snahou
je vytvoriť algoritmus, ktorý rieši problém s najlepšou časovou zložitosťou. Keďže
ide o NP-ťažký problém, skúmajú sa riešenia s exponenciálnou časovou zožitosťou. Pre
zjednodušenie sa používa $O^*$-notácia, ktorá zanedbáva polynomiálne faktory v zložitosti.

Najrýchlejším algoritmom na hľadanie minimálneho rozsahu $L(2,1)$-farbenia grafu
je algoritmus od Junosza-Szaniawskeho a kol., ktorý dosahuje časovú zložitosť
na grafe s $n$ vrcholmi $O^*(2.6488^n)$ \cite{junosza_fast}. Časová zložitosť tohto algoritmu je úzko spätá
s kombinatorickou štruktúrou zvanou vlastný pár -- časová zložitosť algoritmu je
zhruba polynomiálne závislá od počtu vlastných párov v grafe.

Na druhú stranu platí, že najviac vlastných párov majú stromy -- grafy, na ktorých
vieme problém riešiť v polynomiálnom čase. V tejto práci sa budeme venovať využitiu
metódy rozdeľuj a panuj na konštrukciu rýchlejšieho algoritmu.

V druhej kapitole sa pozrieme na delenie problému na nezávislé časti s použitím
hranového separátora. Podrobnejšie sa pozrieme na grafy s mostami a pre dostatočne
veľké $k$ ukážeme, ako zredukovať problém existencie $L(2,1)$-farbenia s rozpätím $k$
v ľubovoľnom grafe na problém na triede grafov, v ktorých odstránením ľubovoľného mostu vznikne izolovaný
vrchol. Z rýchleho riešenia na tejto špeciálnej triede grafov potom bude vyplývať
rýchle riešenie pre všeobecné grafy.

V 3. kapitole ukážeme algoritmus, ktorý nadviaže na prácu Junosza-Szaniawekeho a kol.
Pomocou delenia problému skonštruujeme algoritmus pre planárne grafy s časovou
zložitosťou $O^*(2.2^{n+ o(n)})$ a algoritmus pre dobre rozdeliteľné grafy s časovou
zložitosťou $O^*(2.613^n)$.

Vo 4. kapitole odprezentujeme experiment na triede $2$-hranovo súvislých grafov.
Ukážeme, že pri grafoch do $20$ vrcholov majú najväčší počet vlastných párov
práve kružnice. Zaujímavým výsledkom tejto časti je aj postup generovania všetkých
neoznačených $2$-hranovo súvislých grafov (s opakovaním).
